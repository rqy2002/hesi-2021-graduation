\subsection{Cuspidal automorphic forms}

Automorphic forms are generalizations of modular forms. Here we give the definition and some properties briefly.

\begin{definition}[\textbf{Automorphic forms with central quasi-character $\omega$}]
Let $\omega$ be a character of $ \mathbb Q^*\backslash  \mathbb I$. An automorphic form of $G(\mathbb A)$ with central character $\omega$ is a complex valued function $\phi$ on $G(\mathbb A)$ satisfying:
\begin{enumerate}
\item  $\phi\left(\left (\begin{array}{cc}
z & 0\\
0 & z
\end{array} \right )g\right )=\omega(z)\phi(g)$ for any $z\in \mathbb I$, $g\in G(\mathbb A)$;
\item  $\phi(\gamma g)=\phi(g)$ for any $\gamma\in G(\mathbb Q)$, $g\in G(\mathbb A)$.\\
\item  (\textbf{Smoothness}) For every $g\in G(\mathbb A)$, there is a neighborhood $N$ of $g$ and a smooth function $\phi_\infty$ on $G(\mathbb R)$ such that $\phi(h)=\phi_\infty(h_\infty)$ for $h\in N$.
\item  (\textbf{$K$-finiteness}) $\{\phi(\cdot k); k\in K\}$ spans a finite-dimensional space.
\item  (\textbf{$Z(U(\fr g))$-finiteness}) $\{D\phi; D\in Z(U(\fr g))\}$ spans a finite-dimensional space.
\item  (\textbf{Moderate growth}) There exist constants $C$ and $N$ such that
\[\left |\phi(g) \right |< C\prod _v \mathrm {max}\left \{ \left |g_{ij,v}\right |_v, \left |\mathrm {det} (g_v) \right |^{-1}_v\right \}^N.\]
\end{enumerate}
We denote the space of automorphic forms with central character $\omega$ by $\mathcal A(\omega)$.
\end{definition}

\begin{definition}[\textbf{Cuspidal automorphic forms}]\label{CAF}
An automorphic form $\phi$ is called cuspidal if
\begin{equation}\label{cuspidal}
\int _{\mathbb Q\backslash \mathbb A}\phi\left ( \left (\begin{array}{cc}
1 & a\\
0 & 1
\end{array} \right )g\right )da=0
\end{equation}
for every $g$. We denote the space of cuspidal forms by $\mathcal A_0(\omega)$.
\end{definition}

\begin{definition}[\textbf{$(\fr g,K)\times G(\mathbb A_f)$-modules}]
A $(\fr g,K)\times G(\mathbb A_f)$-module is a complex vector space with actions $\pi_{\fr g}, \pi_{K}$ and
\[\pi_{\mathrm f}\colon G(\mathbb A_f)\to\mathrm{GL}(V),\]
such that $V$ is a  $(\fr g,K)$-module and $\pi_\mathrm f$ is commute with $\pi_{\fr g}, \pi_{K}$.
\end{definition}

\begin{lemma}
Let $K$ and $G(\mathbb A_f)$ act on  $\mathcal A(\omega)$ by right translation and $\fr g$ acts as differential operator at $\infty$. Then $\mathcal A(\omega)$ is a $(\fr g,K)\times G(\mathbb A_f)$-module.
\end{lemma}

\begin{proof}
See [Bum98].
\end{proof}

\begin{definition}[\textbf{Admissible $(\fr g,K)\times G(\mathbb A_f)$-module}]
Let $K_{\mathbb A} =K\cdot K_f$. Then a $(\fr g, K)\times G(\mathbb A_f)$-module $V$ is said admissible if for any irreducible finite representation $\sigma$ of $K$, $V^\sigma$ is of finite dimension. 
\end{definition}

\begin{lemma}\label{61}
$\mathcal A_0(\omega)$ is an admissible $(\fr g,K)\times G(\mathbb A_f)$-module. Moreover, it is a direct sum of irreducible $(\fr g,K)\times G(\mathbb A_f)$-modules.
\end{lemma}

\begin{proof}
See [Bum98].
\end{proof}

\begin{definition}[\textbf{Automorphic representations}]
An automorphic representation is an irreducible $(\fr g,K)\times G(\mathbb A_f)$-module which can be realized by a subquotient of $\mathcal A(\omega)$. An automorphic cuspidal representation is an irreducible $(\fr g,K)\times G(\mathbb A_f)$-module which can be realized by a submodule of $\mathcal A_0(\omega)$ .
\end{definition}
