% !TeX root = ../thuthesis-example.tex

\chapter{Introduction}
Since Yau proved\cite{yau1978ricci} Calabi conjecture, it has always been an important question on constructing the explicit Calabi--Yau metrics. In 2000, Mark Gross and P.M.H.Wilson construct\cite{gross2000large} an asymptotic to Ricci-flat metric on elliptic $K3$ surface, they construct an explicit metric on the smooth part of the fibration, and gluing it with suitable Ouguri--Vafa metric\cite{ooguri1996summing} near the singular fiber. If the elliptic fibers are of area $\epsilon>0$, then the correction term of the approximation to the real Ricci-flat metric is $O(e^{-C/\epsilon})$ for some constant $C>0$.\\ \indent
In various asymptotics towards the goal, ALG and ALH space seems a good attempt. The ALG spaces are asymptotic to the twisted products of flat cones or flat tori, and for ALH spaces, the constructed metrics have an isometrically split cross section $S^1\times T^2$.\\ \indent
Hein gives\cite{hein2012gravitational} some new families of gravitational instantons which helps to the issues of these spaces. In his paper, he describes sets of ALG and ALH spaces, and discuss the volume growth and injective radius decay of the metrics. And this paper is aimed to understand the construction of his metric, in order to better understand how the approximation works.\\ \indent
The main thought is as follows: for a rational elliptic surface, we first classifies its singular fibers with the help of Kodaira's work, and construct the semi-flat metric near the fiber. We require the metric has a proper expression so that we can easily discuss its geometric properties like whether they satisfies the SOB and CYL condition. After that, we try to glue this metric with the original one through complicated Sobolev inequalities, and finally deduce some existence result.\\ \indent
Chapter 2 reviews the basic properties of Weirstrass model, since it is birational to the general elliptic fibration case. Chapter 3 discuss the metric on complex torus bundle, since we want to have a look at the  local behavior on smooth part of the fibration. Chapter 4 then gives the description near singular fibers, and in chapter 5, we construct the semi-flat metric near singular fibers and see the aymptotic behaviors, finally we give an existence result for certain Calabi--Yau metric.


