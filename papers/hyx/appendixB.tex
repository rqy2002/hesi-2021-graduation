\newtheorem{thmb}{Theorem B}

\hspace*{2em}%
本文的主要内容是Maryam Mirzakhani的两篇文章\cite{Mirzakhani:2006fta}和\cite{growthofsimple}。主要的结果是用新的方法证明黎曼曲面的模空间对应的Weil--Petersson体积是一个关于曲面的边界长度的多项式,以及对于给定曲面,它上面的长度小于$L$的简单闭测地线的数量随着$L$趋于无穷具有$L$的多项式的渐进形式。

给定一个亏格为$g$的闭双曲黎曼曲面$S$,它对应的Teichm\"uller空间$\mathscr{T}(S)$是一些映射的等价类构成的集合, 这些映射是形如$f:S\to X$ 的微分同胚,并且两个映射$f:S\to X$与$g:S\to Y$等价意味着$g\circ f^{-1}:X\to Y$ 同伦或者同痕于一个共形映射。 

如果$S_{g,n}$是带有边界的,它的$n$个边界$\{\beta_1,\cdots,\beta_n\}$的长度是$L=(L_1,\cdots,L_n)$,那么Teichm\"uller 空间中涉及的曲面$X$的边界要求是给定的长度$L.$ 对应的Teichm\"uller 空间记为$\mathscr{T}_{g,n}(L)$。

对于一个双曲曲面,它的每个道路同伦等价类中都含有唯一一条测地线,这条测地线是同伦等价类中的曲线中最短的,并且可以用任意一串逼近这个最小值的曲线来取子列去一致逼近。

对于一个曲面,带有$n$条边界,亏格为$n$,可以用$3g-3+n$条不相交的简单闭曲线把曲面分成$2g-2+n$个区域,每个区域同胚于$S_{0,3}$,叫它pants。因为上述原因,它们可以取为不同伦于边界测地线的简单闭测地线。这样子的一个曲线簇$P$即为一个pants分解。每个$S_{0,3}$的复结构可以由三条边界的长度唯一确定。沿着每条曲线,可以用一个一维参数刻画两个区域之间是如何连接的。这给出了$\mathscr{T}_{g,n}(L)$的Fenchel--Nielsen坐标:
\begin{align*}
\mathscr{T}_{g,n}(L)\to & \mathbb{R}_+^P\times \mathbb{R}^P\\
X\to (l_{\alpha_i}&(X),\tau_{\alpha_i}(X)). 
\end{align*}
因此,Teichm\"uller 空间是一个6g-6+2n维的空间。

映射类群$\Mod_{g,n}$定义为$S_{g,n}$的保持边界的保持定向的微分同胚的同伦等价类,它在Teichm\"uller 空间上有一个群作用,对于$h\in \Mod_{g,n}$,它作用到$(X,f)$上得到$(X,f\circ h^{-1})$。这个群作用是除了恒等映射之外没有不动点的,并且是纯不连续的。这个作用对应的商空间就是黎曼曲面$S_{g,n}$对应的模空间$\mathscr{M}_{g,n}(L)$。

映射类群中包含一些特殊的元素。如果$\alpha$是$P$中的一条曲线,可以沿着它对应的一个衣领(collar)邻域,切开后扭转一圈,然后重新连起来,其它部分保持不变,这个映射$D(\alpha)$叫做沿着$\alpha$的Dehn扭曲。

Teichm\"uller空间上面带有一个辛结构Weil-Petersson形式,根据 Scott Wolpert的结果\cite{wolpertformula},这个辛结构用Fenchel-Nielsen坐标的表示是
$$
\omega=\sum_{i=1}^{3g-3+n}dl_i\wedge d\tau_i.
$$
辛结构$\omega$自然诱导了一个体积形式$$
\frac{\wedge^k\omega}{k!}, k=3g-3+n.
$$

通过Wolpert的公式,这两个微分形式在映射类群的作用下可以保持不变,所以可以诱导模空间上面的微分形式。Mirzakhani在\cite{Mirzakhani:2006fta}中证明了, 模空间对应的Weil-Petersson体积$\mathrm{Vol}(\mathscr{M}_{g,n}(L))$是关于边界长度$L$的多项式。

黎曼曲面$X$上的多重曲线形如$\gamma=\sum_{i}^kc_i\gamma_i$,其中$c_i$大于零,$\gamma_i$是简单闭测线,并且互不相交。多重曲线的长度定义为$$
l_\gamma(X)=\sum_{i}c_il_{\gamma_i}(X).
$$
对于一个函数$f:\mathbb{R}\to \mathbb{R}$,它可以诱导模空间上面的函数$$
f_\gamma(X)=\sum_{[\alpha]\in \Mod\cdot [\gamma]}f(l_\alpha(X)).
$$
因为下标取遍$\gamma$在映射类群下作用的整个轨道,因此它在模空间上面良好定义。

Mirzakhani给出了$f_\gamma$在模空间上面的积分公式。

\begin{thmb}[Mirzakhani积分公式]
如果$\Gamma=\{\gamma_1,\cdots,\gamma_k\}$是出现在多重曲线$\gamma$中的所有简单闭测地线,如果$(g,n)\neq (1,1),(0,3)$,那么对于可积函数$f$,有公式成立:
 $$
 \int_{\mathscr{M}_{g,n}(L)}f_\gamma(X)dX=\frac{2^{-M(\gamma)}}{|sym(\gamma)|}\int_{\mathbb{R}_+^k}f(|x|)\mathrm{Vol}(\mathscr{M}_{g,n}(\Gamma,x,\beta,L))x\cdot dx.
 $$

\end{thmb}

这里$|x|$表示$\sum_{i=1}^kc_ix_i$, $x\cdot dx$表示$x_1\cdots x_ndx_1\cdots dx_n$, $\sym(\gamma)$是$\gamma$的对称群,它是依赖于系数的,严格的定义是$$
  \sym(\gamma)=\stab(\gamma)/\cap_{i=1}^k \stab(\gamma_i),
  $$
其中,$\stab(\alpha)$代表$\alpha$的不动子群, $$
 \stab(\gamma)=\{g\in \Mod_{g,n}|g\gamma=\gamma\}.
 $$
 $M(\gamma)$代表$S_{g,n}$沿着$\gamma$切开后出现的$S_{1,1}$的份数,它出现的原因是$S_{1,1}$上面会有一个half 扭曲,不是通过Dehn扭曲来实现的。
 
$\mathscr{M}_{g,n}(\Gamma,x,\beta,L)$以$\mathscr{M}(S_{g,n}(\gamma),x,L)$作为覆叠空间,本身对应的是那些保持被切开后曲面的相同长度的两条边界对应一致的情况。

Mirzakhani通过考虑长度映射的不同水平集的体积来证明了这个公式,但是在\cite{Mirzakhani:2006fta}最开始的公式中系数是有小误差的,在之后的文章中这个公式已经被修正了,Petri也总结了这些事情\cite{Teithe}。

Greg MaShane在\cite{McShane1998SimpleGA}中证明了 McShane恒等式,它是对于带有尖(cusp)的曲面而言的,带有尖的曲面可以类比于边界测地线的长度是零之后做一下紧化。

\begin{thmb}[McShane恒等式]
对于所有无序的简单闭测地线对$(\alpha,\beta)$,满足它们与某个特定的尖一起围成一个pants,有恒等式$$
\sum_{\alpha,\beta}\frac{1}{1+\exp{\frac{l_\alpha(X)+l_\beta(X)}{2}}}=\frac{1}{2}
$$成立。
\end{thmb}


Mirzakhani推广了这个式子。

\begin{thmb}[McShane--Mirzakhani 恒等式]
对于任意的双曲曲面$X\in \mathscr{T}_{g,n}(L)$,有$$
\sum_{\{\gamma_1,\gamma_2\}\in\mathscr{F}_1}\mathscr{D}(L_1,l_{\gamma_1}(X),l_{\gamma_2}(X))+\sum_{i=2}^n\sum_{\gamma\in \mathscr{F}_{1,i}}\mathscr{R}(L_1,L_i,l_{\gamma}(X))=L_1.
$$
\end{thmb}
这里 $F_1$中的元素是所有无序的非边界简单闭测地线对$(\alpha,\beta)$,满足它们与边界$\beta_1$一起围成一个pants,$F_{1,i}$包含所有与$\beta_1,\beta_i$一起围成一个pants的非边界简单闭测地线。

$D,R$是包含三个正数作为变量的函数,数值上面的定义为
\begin{equation*}\label{D}
  D(x_1,x_2,x_3)=2\log \left(\frac{e^{\frac{x}{2}}+e^{\frac{y+z}{2}}}{e^{\frac{-x}{2}}+e^{\frac{y+z}{2}}}\right),
  \end{equation*}
 \begin{equation*}\label{R}
     R(x,y,z)=x-\log \left(\frac{\cosh(\frac{y}{2})+\cosh (\frac{x+z}{2})}{\cosh(\frac{y}{2})+\cosh (\frac{x-z}{2})}\right).
 \end{equation*}
对于一个三条边界长为$x,y,z$的pants,$R(x,y,z)$和$D(x,y,z)$对应的是长度为$x$的一条边界上面的一段特殊的弧线的长。
Mirzakhani的证明方法就是通过考虑把边界$\beta_1$去掉一个零测集$E_i$之后分成一些特殊的弧线之和,计算每段弧的长度得到这个公式。这个$E_i$是一个康托集和可数集的并。


具体而言,这个零测集$E_i$是边界$\beta_i$上面的一些点,从这些点垂直出发的测地线都是不自交的,并且和边界相交的时候都是垂直的。根据\cite{BIRMAN1985217},它首先是零测的,其次在$\beta_i$中的拓扑是很特别的。分类这些点是MaShane的工作\cite{McShane1998SimpleGA}。

Mirzakhani把这个恒等式在模空间上面积分,由此得到了一个关于模空间体积的递推公式,据此模空间的体积是边界长度的多项式,并且这个多项式的次数和系数具有比较好的限制。
\begin{thmb}
$$\mathrm{Vol}_{wp}(\mathscr{M}_{g,n}(L))=\sum_{|\alpha|\leq 3g-3+n}C_\alpha\cdot L^{2\alpha}.$$

其中 $L^{2\alpha}$表示 $L_1^{2\alpha_1}\cdots L_k^{2\alpha_k}$, 并且系数满足 $C_\alpha\in \pi^{6g-6+2n-2|\alpha|}\cdot \mathbb{Q}$.
\end{thmb}

在\cite{growthofsimple}中Mirzakhani用同样的积分公式去计数了长度小于$L$的简单闭测地线的数目。

如果不要求测地线是不自交的,那么这个数量随着长度发散到无穷具有极限形式$\frac{e^L}{L}$。如果只考虑简单闭测地线,那么根据Mirzakhani的结果,这个数量具有极限形式$c(X)\cdot L^{6g-6+2n}$,前面的系数依赖于曲面在模空间中的位置。

证明的过程需要用到测地叠层(geodesic lamination)作为工具,大致上而言,一个叠层是由黎曼曲面上面一些不交的曲线的并围成的闭集,而测地叠层要求这些曲线是测地线。特别地,有限条不交的简单闭测地线是一个测地叠层。一个紧支的测地叠层可以带有类似于测度的结构,这个结构大致上刻画曲线穿过了叠层中的多少层。


带有测度的测地叠层构成一个空间,这个空间用符号$\mathscr{ML}_{g,n}$表示。它具有分段线性的结构,维数是$6g-6+2n$。它的整点和系数为整数的多重曲线一一对应,其中的整点集合为$\mathscr{ML}_{g,n}(\mathbb(Z))$。这个空间上面有一个Thurston测度$\mu_{Th}$,它在映射类群的作用下保持不变。Masur用egrodic理论在\cite{ergodic}中证明了某种程度上这种测度是唯一的。

定义$\mathscr{ML}_{g,n}(\mathbb(Z))$中的整点,满足对应的多重曲线的长度小于$L$的个数,记为$b_X(L)$。

定义$\mathscr{ML}_{g,n}$中长度小于$1$的球为$B_X$,那么根据$\mathscr{ML}_{g,n}$的分段线性结构直接就可以有渐进行为$$
\frac{b_X(L)}{L^{6g-6+2n}}\to \mu_{Th}(B_X),L\to \infty.
$$
对于这个极限函数的可积性,是通过估计$b_X(L)$,然后应用控制收敛定理得到的。用组合的工具给出多重曲线的另外一种长度的定义,

$$
L_P(X,\gamma)=\sum_{\alpha\in P} (I(\gamma,\alpha)w(\alpha)+|tw(\gamma,\alpha)|l(\alpha)),
$$
其中$I$是相交数,刻画和pants分解中曲线的交点,$tw$是扭曲数,刻画沿着pants分解中曲线环绕的圈数。

当曲面对应的pants分解中的曲线长度都有限(小于等于L)的时候,用一些双曲几何的工具可以证明这个长度和原始定义的长度是相容的,即存在常数$c(L,g,n)$使得$$
\frac{1}{c(L,g,n)}l_\gamma(X)\leq L_{P}(X,\gamma)\leq c(L,g,n) l_\gamma(X).
$$

而因为$L_P(X,\gamma)$是一些系数的线性组合,它的范围是比较好估计的,一个足够的估计是

\begin{thmb}
对于足够小的 $\epsilon>0$,如果 $L>1$, 就存在与$X$无关的常数$C_1,C_2$使得$$
C_1 \prod_{\gamma:l_{\gamma}(X)\leq \epsilon}\frac{1}{l_{\gamma}(X)|\ln l_{\gamma}(X) |}\leq  \frac{b_X(L)}{L^{6g-6+2n}}\leq C_2\prod_{\gamma:l_{\gamma}(X)\leq \epsilon}\frac{1}{l_{\gamma}(X)}.
$$
\end{thmb}

据此可以得到$\mu_{Th}(B_X)$与$\frac{b_X(L)}{L^{6g-6+2n}}$都是在模空间上面可积的函数。其中条件pants分解中的曲线长度有限,可以用Bers常数一致地控制住。

通过把$[0,L]$上的特征函数$f$运用到Mirzakhani的积分公式上去,可以得到
$$
\int_{\mathscr{M}_{g,n}}S_X(L,\gamma)dX=\frac{2^{-M(\gamma)}}{|\sym(\gamma)|}\int_{\sum_{i}a_ix_i\leq L}V_{g,n}(\Gamma,x)xdx,
$$
得到的结果是一个次数为$6g-6+2n$的关于$L$的多项式,其中$S_X(L,\gamma)$表示$\gamma$在映射类群作用下的轨道中,长度小于$L$的数量。

如果把它的首项系数记为$c(\gamma)$,那么最终的结论是,对于首项系数为有理数的多重曲线,$S_X(L,\gamma)$具有多项式的渐进形式。

\begin{thmb}
对于有理多重曲线,$\gamma\in \mathscr{ML}_{g,n}$, $$
S_X(L,\gamma)\sim \frac{\mu_{Th}(B_X)\cdot c(\gamma)}{b_{g,n}} L^{6g-6+2n}.
$$
\end{thmb}

这里$b_{g,n}$是$\mu_{Th}(B_X)$在模空间上面的积分。

这个定理的证明是用计数的方法给出$\mathscr{ML}_{g,n}$上面的一串测度,并且这串测度的任意子列都有进一步的子列弱收敛到Thurston测度的同一个倍数,因而整个串测度收敛到Thurston测度的倍数。

具体而言,给定$\gamma$和$T>0$,可以定义测度 $$
 \mu_{T,\gamma}(U)=\frac{\#\{T\cdot U\cap \Mod_{g,n}\cdot \gamma \}}{T^{6g-6+2n}}.
 $$
 随着$T\to\infty$这是一串测度,对它的子列可以进行上面这些操作。对于每个收敛的子列的弱极限测度,
 验证它关于Thurston测度是连续的,并且在映射类群的下不变,根据Masur的工作,它必然是Thurston测度的倍数。重新在模空间上面积分后可以还原出具体的倍数。
 
 
 另外一方面,根据$\mu_{T,\gamma}$测度的定义,作用到$B_X$上之后,有$\mu_{T,\gamma}(B_X)=\frac{S_{X}(T,\gamma)}{T^{6g-6+2n}}\to v\cdot \mu_{Th}(B_X)$.
 
 这个过程只能对有理系数的多重曲线去做,这是因为有理系数的多重曲线本质上和整系数的多重曲线一致,而整的多重曲线可以一致地用$b_X(L)$控制,但是对于一般的多重曲线得不到这样子一致的上界,无法运用控制收敛定理。
 
 关于具体的系数$c(\gamma)$可以用模空间的体积多项式中最高项的系数去表示,并且和曲面的相交理论有关系,这方面的具体计算还有很多不清楚的地方。