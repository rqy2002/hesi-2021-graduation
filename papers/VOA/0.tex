Quantum field theory has proved to be
a significant source of ideas in contemporary mathematics,
providing insights to many important mathematical problems,
and these insights often become key to solving these problems.

One famous example is the introduction of
vertex algebras by Richard Borcherds \cite{borcherds86},
which are algebraic structures that
come from conformal field theory in physics.
Vertex algebras led to an explanation of a miraculous connection
between the monster simple group and elliptic modular functions,
known as the \emph{moonshine conjecture},
partially solved by Frenkel, Lepowsky and Meurman \cite{flm},
and finally proved by Borcherds in 1992 \cite{borcherds92}.
Since then, vertex algebras has become
important mathematical objects in their own right.

From the physical point of view,
vertex algebras are algebraic structures
that describe the local data of a
two-dimensional chiral conformal field theory,
with the state--field correspondence taken as an axiom.
This information determines the
singular behaviour of $n$-point functions
when two of the $n$ points collide,
known as the \emph{operator product expansion},
or the \emph{OPE}, of the field insertion operators,
which is of great interest in theoretical physics.

From the mathematical point of view,
vertex algebras can be viewed as a two-dimensional analogue
of associative algebras, the latter being one-dimensional algebraic structures.
Indeed, multiplication of $n$ elements in an associative algebra
can be viewed as placing $n$ points on the real line
without overlapping them,
and the order of multiplication is determined by the
position of these $n$ points.
On the other hand, multiplication of $n$ elements in a vertex algebra
can be viewed as placing $n$ points on the complex plane
(more precisely, the formal disk),
and the multiplication operation depends holomorphically
on the position of these $n$ points.

Keeping in mind the parallelism between associative and vertex algebras,
in vertex algebra theory, one may consider analogous concepts of
many constructions in associative algebra theory.
For example, representations of associative algebras
are analogous to representations of vertex algebras;
characters of an associative algebra
are analogous to elements of the \emph{conformal block} of a vertex algebra,
which also serve as candidates of the correlation function
of the conformal field theory,
describing its global data.

For associative algebras,
characters often come from taking the trace in a representation.
Similarly, for vertex algebras,
under some assumptions,
elements of the conformal block come from the trace in a representation as well,
as is shown in the work of Zhu \cite{zhu}.

In this paper, we consider conformal field theories
that arise as chiral deformations of other theories.
From the vertex algebra point of view,
this corresponds to the BRST reduction of a vertex algebra,
which is a way of obtaining new vertex algebras from old ones.
One famous example is the $\scrW$-algebra,
which is the BRST reduction of the Kac--Moody vertex algebra.

We propose a method of constructing families of elements
of the genus $1$ conformal block
for the BRST reduction of a vertex algebra,
also called the \emph{BRST twisted conformal block},
using the technique of regularized integrals,
recently developed by Li and Zhou \cite{regularized}.
Our main result (Theorem~\ref{thm-main}) shows that
given an element of the conformal block for the original vertex algebra,
its regularized integral against the BRST operator
gives an element of the conformal block
for the BRST reduction.

This paper is organized as follows.
In section 2, we review the definition of a vertex algebra,
and present some basic properties and constructions.
In section 3, we discuss two examples of BRST reduction.
In section 4, we give a brief introduction to Li and Zhou's technique
of the regularized integral, and we prove the main result of this paper (Theorem~\ref{thm-main}),
which produces a family of elements of the BRST twisted conformal block.
