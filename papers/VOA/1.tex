% \subsection{Conformal field theory}

% A \emph{conformal field theory} in two dimensions
% is a two dimensional quantum field theory that is
% invariant under conformal transformations.

% Consider a conformal field theory on
% the space-time cylinder $S^1 \times \bbR$,
% where $S^1$ is the space component and $\bbR$ is the time component.
% We use the complex coordinate $\xi = \theta + \upi \tau$,
% where $\theta \in S^1$ and $\tau \in \bbR$.
% We then change the coordinates to
% \[
%     z = \upe^{-\upi \xi}.
% \]
% This maps the cylinder to the set $\bbC \setminus \{ 0 \}$,
% and the time ordering becomes the radial ordering
% of concentric circles centred at the origin.

% \figplacehold


\subsection{Definition}

First of all,
we briefly recall the definition of a vertex algebra.
For details, we refer the reader to
\cite{frenkel-ben-zvi} and \cite{kac}.

\begin{definition}
    A \emph{vertex algebra} is the data $(\scrV, Y, T, \ket{0})$, where
    \begin{itemize}
        \item
            $\scrV$ is a $\bbZ_2$-graded vector space over $\bbC$,
            called the \emph{state space},
            or the \emph{space of observables}.
        \item
            $\ket{0} \in \scrV$ is an even element,
            called the \emph{vacuum state}.
        \item
            $Y$ is an even linear map
            \begin{align*}
                Y \colon \scrV & \to \End(\scrV) [[z, z^{-1}]], \\
                A & \mapsto A (z) = \sum_{n \in \bbZ} A_{(n)} z^{-n - 1},
            \end{align*}
            called the \emph{state--field correspondence}.
        \item
            $T \colon \scrV \to \scrV$
            is an even endomorphism,
            called the \emph{translation operator}.
    \end{itemize}
    They satisfy the following axioms:
    \begin{itemize}
        \item
            For all $A, B \in \scrV$, we have
            \[
                A (z) \, B \in \scrV ((z)),
            \]
            where $\scrV ((z)) = \scrV [[z]] [z^{-1}]$
            is the space of formal Laurent series
            with values in $\scrV$.
            In other words, we have
            \[
                A_{(n)} B = 0 \quad (n \gg 0).
            \]
            
        \item
            (Vacuum)
            $T \ket{0} = 0$
            and $\ket{0} (z) = \operatorname{id}$.
            For all $A \in \scrV$,
            $A (z) \ket{0} = A + O (z)$.
            
        \item
            (Translation)
            For all $A \in \scrV$, we have $[T, A(z)] = \partial_z A(z)$.
            
        \item
            (Locality)
            For all $A, B \in \scrV$, there exists $N \in \bbN$ such that
            \[
                (z - w)^N [A(z), B(w)] = 0.
            \]
    \end{itemize}
    By abuse of language,
    we say that $\scrV$ is a vertex algebra.
    \varqed
\end{definition}

For $A \in \scrV$, note that the operator $A_{(n)} \in \End (\scrV)$
can be expressed as a residue:
\[
    A_{(n)} = \operatorname{Res}_{z = 0} (z^n A(z)).
\]

\begin{definition}
    A \emph{graded vertex algebra}
    is a vertex algebra $\scrV$,
    together with a $\bbZ$-grading
    \[
        \scrV = \bigoplus_{m} \scrV_m
    \]
    on the vector space $\scrV$, compatible with the $\bbZ_2$-grading
    (i.e.\ the two gradings form a $\bbZ \times \bbZ_2$-bigrading),
    such that
    \begin{itemize}
        \item
            $\ket{0} \in \scrV_0$.
        \item
            $\deg T = 1$.
        \item
            For any $A \in \scrV_m$,
            we have $\deg A_{(n)} = -n - 1 + m$.
    \end{itemize}
    If $A \in \scrV_m$,
    we say that $A$ has \emph{conformal dimension} $m$.
    \varqed
\end{definition}

\begin{definition}
    A \emph{conformal vertex algebra},
    also called a \emph{vertex operator algebra},
    is a graded vertex algebra $\scrV$,
    together with an even element $L \in \scrV_2$,
    called the \emph{conformal vector},
    such that if we write
    \[
        L (z) = \sum _{n \in \bbZ} L_{n} z^{-n-2}
    \]
    (that is, $L_n = L_{(n+1)}$, so that $\deg L_n = -n$), then
    \begin{itemize}
        \item
            $L_{-1} = T$ is the translation operator.
        \item
            For all $m$, $\scrV_m$ is the $m$-eigenspace of $L_0$.
        \item
            We have the \emph{Virasoro relations}
            \[
                [L_n, L_m] =
                (n - m) L_{n + m} +
                c \cdot \frac{n^3 - n}{12} \cdot \delta_{n, -m},
            \]
            where $c \in \bbC$ is called the \emph{central charge}.
            \varqed
    \end{itemize}
\end{definition}


\subsection{Operator product expansion}

The analogue between associative algebras and vertex algebras
can be formulated as follows.
For an associative algebra $V$, each element $A \in V$
defines an element of $\End (V)$ by left multiplication,
so the structure of the algebra is given by a linear map
\begin{align*}
    Y \colon V & \to \End (V), \\
    A & \mapsto A \cdot {-}.
\end{align*}
In a vertex algebra $\scrV$, there is a family of multiplication operators,
parametrized by the formal variable $z$,
and the structure of the vertex algebra is given by
\begin{align*}
    Y \colon \scrV & \to \End(\scrV) [[z, z^{-1}]], \\
    A & \mapsto A (z) \cdot -.
\end{align*}

From this point of view,
one can talk about whether a vertex algebra
is associative.
It turns out that all vertex algebras
are associative in the following sense.

\begin{theorem}
    \label{thm-assoc}
    Let $\scrV$ be a vertex algebra,
    and let $A, B, C \in \scrV$. 
    \begin{itemize}
        \item
            \textup{(Associativity)}
            The elements
            \begin{align*}
                A (z) \, B (w) \, C
                & \in \scrV ((z)) ((w)), \\
                B (w) \, A (z) \, C
                & \in \scrV ((w)) ((z)), \\
                (A (z - w) \, B) (w) \, C
                & \in \scrV ((w)) ((z - w))
            \end{align*}
            come from the same element of
            \[
                \scrV [[z, w]] [z^{-1}, w^{-1}, (z-w)^{-1}],
            \]
            where the inclusions
            \begin{align*}
                \scrV [[z, w]] [z^{-1}, w^{-1}, (z-w)^{-1}]
                & \hookrightarrow \scrV ((z)) ((w)), \\
                \scrV [[z, w]] [z^{-1}, w^{-1}, (z-w)^{-1}]
                & \hookrightarrow \scrV ((w)) ((z)), \\
                \scrV [[z, w]] [z^{-1}, w^{-1}, (z-w)^{-1}]
                & \hookrightarrow \scrV ((w)) ((z-w))
            \end{align*}
            are given by
            \begin{align*}
                (z-w)^{-1} & \mapsto
                \frac{1}{z} \sum_{n=0}^\infty
                \biggl( \frac{w}{z} \biggr)^n, \\
                (z-w)^{-1} & \mapsto
                -\frac{1}{w} \sum_{n=0}^\infty
                \biggl( \frac{z}{w} \biggr)^n, \\
                z^{-1} & \mapsto
                \frac{1}{w} \sum_{n=0}^\infty
                \biggl( \frac{w-z}{w} \biggr)^n,
            \end{align*}
            respectively.
    \end{itemize}
\end{theorem}

See \cite[Theorem~3.2.1]{frenkel-ben-zvi}.

As a result,
we may write down equations of the form
\[
    A (z) \, B (w) =
    \sum_{n \in \bbZ} \frac{(A_{(n)} B) (w)}{(z - w)^{n+1}},
\]
or sometimes, we only write down the singular part:
\[
    A (z) \, B (w) =
    \sum_{n = 0}^N \frac{(A_{(n)} B) (w)}{(z - w)^{n+1}}
    + \text{reg.},
\]
where `reg.'\ means `regular part'.
These formulas are known as the
\emph{operator product expansion}, or the \emph{OPE},
of the fields $A$ and $B$
\cite[\S3.3]{frenkel-ben-zvi}.

Note that technically,
the OPE formulas only make sense in the space
\[
    \scrV [[z, w]] [z^{-1}, w^{-1}, (z-w)^{-1}].
\]

From the OPE, we can recover the commutation relations
of vertex operators.
Indeed, for even elements $A, B \in \scrV$, we have
\begin{align*}
    \oint_w A (z) \, B (w) \, d z
    & = \oint_{\gamma_1} A (z) \, B (w) \, d z
    - \oint_{\gamma_2} B (w) \, A (z) \, d z \\
    & = \Biggl[ \ \oint A \ , B (w) \Biggr],
\end{align*}
where $\oint_w$ denotes the contour integral
along a small circle around $w$;
$\gamma_1, \gamma_2$ are as shown in the following picture;
and $\oint A$ is short for $\oint_0 A (z) \, d z$,
and is equal to $2 \uppi \upi \, A_{(0)}$.

\[
    \begin{tikzpicture}[
        thick,
        decoration = {
            markings,
            mark = at position 0.05 with {\arrow[scale=1.5]{stealth}}
        },
        scale = .8
    ]
        \begin{scope}
            \fill (0, 0) circle (.07);
            \fill (1.8, 1.8) circle (.07);
            \draw[postaction = {decorate}] (1.8, 1.8) circle (.6);
            \draw[postaction = {decorate}] (0, 0) circle (3.6);
            \draw[postaction = {decorate}] (0, 0) circle (1.4);
            \node at (.25, -.15) {$0$};
            \node at (2.05, 1.65) {$w$};
            \node at (1.7, .1) {$\gamma_2$};
            \node at (3.8, .8) {$\gamma_1$};
        \end{scope}
    \end{tikzpicture}
\]

Integrating over $w$, we obtain the formula
for the commutator of $\oint A$ and $\oint B$:
\[
    \Biggl[ \ \oint A \ , \ \oint B \Biggr] =
    \oint_0 \ d w \ 
    \oint_w \ d z \ A (z) \, B (w).
\]
See also \cite[Ch.~6]{cft}.
A similar argument shows that this holds for odd elements as well.

We can slightly generalize this argument,
by replacing the integrand $A (z) \, B (w)$
by $z^m w^k \, A (z) \, B (w)$.
Then one arrives at the following result,
a version of which was taken as
one of the axioms in Borcherd's original definition
\cite{borcherds86} of a vertex algebra.

\begin{theorem} [Borcherd's identity]
    For any $A, B \in \scrV$ and $m, k \in \bbZ$, we have
    \[
        [A_{(m)}, B_{(k)}] =
        \sum_{n \geq 0} \binom{m}{n} \,
        (A_{(n)} B)_{(m+k-n)}.
    \]
    In particular, when $m = 0$, we have
    \[
        [A_{(0)}, B_{(k)}] =
        (A_{(0)} B)_{(k)}.
    \]
\end{theorem}

See also \cite[\S3.3]{frenkel-ben-zvi}
for a generalization.

\begin{proof}
    \allowdisplaybreaks
    We have
    \begin{align*}
        [A_{(m)}, B_{(k)}]
        & = \frac {1} {(2 \uppi \upi)^2}
        \Biggl[ \ 
            \oint_0 \ z^m A (z) \, d z \ , \ 
            \oint_0 \ w^k B (w) \, d w
        \Biggr] \\
        & = \frac {1} {(2 \uppi \upi)^2}
        \oint_0 \ w^k \, d w \ 
        \oint_w z^m \, d z \, A (z) \, B (w) \\
        & = \frac {1} {(2 \uppi \upi)^2}
        \oint_0 \ w^k \, d w \sum_{n \in \bbZ} \ 
        \oint_w \frac{z^m}{(z - w)^{n+1}} \, d z \,
        (A_{(n)} B) (w) \\
        & = \frac {1} {2 \uppi \upi}
        \oint_0 \ w^k \, d w \sum_{n \geq 0} \,
        \binom{m}{n} \, w^{m-n} \,
        (A_{(n)} B) (w) \\
        & = \sum_{n \geq 0} \binom{m}{n} \,
        (A_{(n)} B)_{(m+k-n)}. \qedhere
    \end{align*}
\end{proof}

Note that only the singular part of the OPE $A (z) \, B (w)$
is used in this identity.

On the other hand,
the non-singular part of the OPE
is equal to the normally ordered product,
which we now define following \cite[\S2.2]{frenkel-ben-zvi}.

\begin{definition}
    Let $\scrV$ be a vertex algebra, and let $A, B \in \scrV$.
    The \emph{normally ordered product} of the fields
    $A (z)$ and $B (w)$ is defined by the formula
    \[
        \nord{A (z) \, B (w)} =
        A (z)_+ \, B (w) + B (w) \, A (z)_-,
    \]
    where
    \[
        A (z)_+ = \sum_{n < 0} A_{(n)} z^{-n-1}, \quad
        A (z)_- = \sum_{n \geq 0} A_{(n)} z^{-n-1},
    \]
    where the plus/minus signs refer to the positive/negative powers of $z$.
    
    More generally, for $A_1, \dotsc, A_n \in \scrV$, with $n > 2$,
    the \emph{normally ordered product} of the fields
    $A_1 (z_1)$, $\dotsc$, $A_n (z_n)$ is defined inductively by
    \[
        \nord{A_1 (z_1) \cdots A_n (z_n)} =
        \nord{A_1 (z_1) \ \bigl( \nord{A_2 (z_2) \cdots A_n (z_n)} \bigr) }.
        \varqedhere
    \]
\end{definition}

The idea behind the normally ordered product is that
by changing the order of multiplication,
we avoid the singularities that we would
normally get in the OPE.
This is shown by the following theorem.

\begin{theorem}
    Let $\scrV$ be a vertex algebra, and let $A, B \in \scrV$. Then
    \[
        A (z) \, B (w) =
        \sum_{n \geq 0} \frac {(A_{(n)} B)(w)} {(z - w)^{n + 1}}
        + \nord{A (z) \, B (w)}.
    \]
    That is, the regular part of the OPE $A (z) \, B (w)$
    is exactly the normally ordered product $\nord{A (z) \, B (w)}$.
\end{theorem}

See \cite[\S3.3.6]{frenkel-ben-zvi}.

Since the normally ordered product $\nord{A (z) \, B (w)}$
is no longer singular along the diagonal $z = w$,
one may consider its restriction to the diagonal,
which is a well-defined field,
denoted by $\nord{A (z) \, B (z)}$.
This works for more than two fields as well.

Finally, we mention a theorem
that describes the field corresponding to
the product of vertex operators.
This result shows that the information of a vertex algebra
can be recovered from its generating elements.

\begin{theorem}
    \label{thm-structure}
    Let $\scrV$ be a vertex algebra, let $A^1, \dotsc, A^k \in \scrV$,
    and let $n_1, \dotsc, n_k \in \bbZ_{<0}$. Then
    \[
        \bigl( A^1_{(n_1)} \cdots A^k_{(n_k)} \ket{0} \bigr) (z) =
        \Biggl( \prod_{i=1}^k \frac{1}{(-n_i - 1)!} \Biggr) \ 
        \nord{ \partial_z^{-n_1 - 1} A^1 (z) \cdots \partial_z^{-n_k - 1} A^k (z) }.
    \]
\end{theorem}

See \cite[Corollary~4.4]{kac}.


\subsection{Examples}
\label{sect-voa-examples}

\begin{example} [$\betagamma$ system]
    \allowdisplaybreaks
    \label{eg-beta-gamma}
    Let 
    \[
        \frg^\betagamma \enspace = \enspace 
        \bigoplus_{n \in \bbZ} \bbC \cdot \beta_n
        \enspace \oplus \enspace \bigoplus_{n \in \bbZ} \bbC \cdot \gamma_n
        \enspace \oplus \enspace \bbC \cdot \mathbf{1}
    \]
    be a Lie algebra defined by
    \[
        [\beta_n, \gamma_m] = \delta_{n, -m} \cdot \mathbf{1}, \quad
        [\beta_n, \beta_m] = [\gamma_n, \gamma_m] = [\mathbf{1}, -] = 0.
    \]
    Let
    \[
        \scrV^\betagamma =
        U (\frg^\betagamma) \cdot \ket{0} \Bigg/
        \left\{ \ 
            \begin{aligned}
                \beta_n \cdot \ket{0} &= 0 \quad (n \geq 0) \\
                \gamma_n \cdot \ket{0} &= 0 \quad (n > 0) \\
                \mathbf{1} \cdot \ket{0} &= \ket{0}
            \end{aligned}
        \ \right\}
    \]
    be a representation of $\frg^\betagamma$
    generated by the vector $\ket{0}$.
    By the PBW theorem,
    as a vector space, one may identify
    \[
        \scrV^\betagamma \simeq
        \bbC [
            \beta_{-1}, \beta_{-2}, \dotsc,
            \gamma_0, \gamma_{-1}, \gamma_{-2}, \dotsc
        ].
    \]
    We assign a $\bbZ$-grading to $\scrV^\betagamma$ by
    \[
        \deg \beta_{n} = -n \ (n < 0), \quad
        \deg \gamma_{n} = -n \ (n \leq 0).
    \]
    This grading is well-defined because the elements
    $\beta_n$ ($n < 0$) and $\gamma_n$ ($n \leq 0$)
    commute with each other in $\frg^\betagamma$.
    
    We wish to define a vertex algebra structure on $\scrV^\betagamma$,
    that is, to define the maps $Y$ and $T$. Write
    \begin{align*}
        \beta (z) &= \sum_{n \in \bbZ} \beta_n z^{-n - 1}, \\
        \gamma (z) &= \sum_{n \in \bbZ} \gamma_n z^{-n}.
    \end{align*}
    In light of Theorem~\ref{thm-structure},
    we define
    \begin{align*}
        \beta_{-n-1} (z) &=
        \frac{1}{n!}
        \biggl( \frac {\partial} {\partial z} \biggr)^n \beta (z), \\
        \gamma_{-n} (z) &=
        \frac{1}{n!}
        \biggl( \frac {\partial} {\partial z} \biggr)^n \gamma (z)
    \end{align*}
    for $n \geq 0$, and for a general element
    $x_1 \cdots x_p \ket{0} \in \scrV^\betagamma$,
    where each $x_i$ is either 
    $\beta_{n}$ ($n < 0$) or $\gamma_{n}$ ($n \leq 0$),
    we define
    \[
        \bigl( x_1 \cdots x_p \ket{0} \bigr) (z) =
        \nord{ x_1 (z) \cdots x_p (z) }
    \]
    to be the normally ordered product.
    
    The commutation relations of $\beta_n$ and $\gamma_n$
    can be rewritten as the OPE
    \[
        \beta(z) \, \gamma(w) = \frac{1}{z - w} + \text{reg.}
    \]
    
    The conformal structure on $\scrV^\betagamma$ is given by the field
    \[
        L (z) = \nord{ \gamma'(z) \, \beta(z) }.
    \]
    One can show that $L$ satisfies the axioms for a conformal vertex algebra,
    and the central charge is $c = 2$. \varqed
\end{example}


\begin{example} [$\bc$ system]
    \label{eg-bc}
    The $\bc$ system is the fermionic counterpart of the $\betagamma$ system.
    Let
    \[
        \frg^\bc \enspace = \enspace 
        \bigoplus_{n \in \bbZ} \bbC \cdot b_n
        \enspace \oplus \enspace \bigoplus_{n \in \bbZ} \bbC \cdot c_n
        \enspace \oplus \enspace \bbC \cdot \mathbf{1}
    \]
    be a Lie superalgebra,
    with all the elements $b_n, c_n$ being odd, and with
    \[
        [b_n, c_m] = \delta_{n, -m} \cdot \mathbf{1}, \quad
        [b_n, b_m] = [c_n, c_m] = [\mathbf{1}, -] = 0.
    \]
    We then copy the construction of the $\betagamma$ vertex algebra
    word by word, with $b_n$ and $c_n$ in place of $\beta_n$ and $\gamma_n$,
    except that these elements are now odd.
    This defines a conformal vertex algebra
    $\scrV^\bc$, with
    \[
        b(z) \, c(w) = \frac{1}{z - w} + \text{reg.},
    \]
    and with the central charge $c = -2$. \varqed
\end{example}


\begin{example} [chiral boson]
    Let
    \[
        \frg^\cb \enspace = \enspace 
        \bigoplus_{n \in \bbZ} \bbC \cdot \phi_n
        \enspace \oplus \enspace \bbC \cdot \mathbf{1}
    \]
    be a Lie algebra (all elements are even), with
    \[
        [\phi_n, \phi_m] = n \delta_{n, -m} \cdot \mathbf{1}, \quad
        [\mathbf{1}, -] = 0.
    \]
    Let
    \[
        \scrV^\cb = 
        U (\frg^\cb) \cdot \ket{0} \Bigg/
        \left\{ \ 
            \begin{aligned}
                \phi_n \cdot \ket{0} &= 0 \quad (n \geq 0) \\
                \mathbf{1} \cdot \ket{0} &= \ket{0}
            \end{aligned}
        \ \right\},
    \]
    so that as a vector space,
    \[
        \scrV^\cb \simeq \bbC[\phi_{-1}, \phi_{-2}, \dotsc].
    \]
    We define the grading on $\scrV^\cb$ by
    \[
        \deg \phi_n = -n \quad (n < 0).
    \]
    Denote
    \[
        \phi (z) = \sum_{n \in \bbZ} \phi_n z^{-n-1},
    \]
    and define the corresponding field by
    \[
        \phi_{-n - 1} (z) =
        \frac{1}{n!}
        \biggl( \frac {\partial} {\partial z} \biggr)^n \phi (z)
    \]
    for all $n \geq 0$.
    For a general element of $\scrV^\cb$,
    the corresponding field is given by the normally ordered product, as before.
    We have the OPE
    \[
        \phi (z) \, \phi (w) = \frac {1} {(z - w)^2} + \text{reg.}
    \]
    
    There is a family of conformal structures on $\scrV^\cb$,
    parametrized by $\lambda \in \bbC$,
    given by
    \[
        L_\lambda = \frac{1}{2} \phi_{-1}^2 + \lambda \phi_{-2},
    \]
    with central charge $c = 1 - 12 \lambda^2$. \varqed
\end{example}


\begin{example} [Kac--Moody vertex algebra]
    \label{eg-kac-moody}
    Let $\frg$ be a finite dimensional simple Lie algebra over $\bbC$.
    The \emph{affine Kac--Moody algebra} associated to $\frg$
    is the Lie algebra whose underlying vector space is
    \[
        \hat{\frg} \enspace = \enspace
        \frg ((t)) \enspace \oplus \enspace \bbC \cdot \mathbf{1},
    \]
    where $\frg ((t)) = \frg \otimes \bbC ((t))$,
    with its Lie bracket defined by $[\mathbf{1}, -] = 0$, and
    \[
        [A \otimes f (t), B \otimes g (t)] =
        [A, B]_{\frg} \otimes f (t) \, g (t) -
        (\Res_{t = 0} f \, d g) \langle A, B \rangle \cdot \mathbf{1},
    \]
    where $A, B \in \frg$ and $f, g \in \bbC ((t))$, and
    \[
        \langle -, - \rangle =
        \frac {1} {2 h^\vee} \cdot (\text{Killing form}),
    \]
    where $h^\vee$ is the \emph{dual Coxeter number} of $\frg$,
    and $1 / 2 h^\vee$ is a normalizing factor,
    so that with respect to the induced inner product on $\frh^\vee$
    (where $\frh$ is the Cartan subalgebra),
    the length of the maximal root of $\frg$ is $\ssqrt{2}$.
    
    Let $k \in \bbC$, and define
    \[
        \scrV_k (\frg) =
        U (\hat{\frg}) \cdot \ket{0} \Bigg/
        \left\{ \ 
            \begin{aligned}
                \frg [[t]] \cdot \ket{0} &= 0 \\
                \mathbf{1} \cdot \ket{0} &= k \ket{0}
            \end{aligned}
        \ \right\}.
    \]
    Let $\{ e^a \}_{a = 1, \dotsc, \dim \frg}$ be a basis of $\frg$,
    and for each $n \leq -1$, let
    \[
        e^a_n = e^a \otimes t^n \in \hat{\frg}.
    \]
    Then as a vector space, $\scrV_k (\frg)$ is spanned by the elements
    \[
        e^{a_1}_{n_1} \cdots e^{a_m}_{n_m} \ket{0},
    \]
    where $a_1 \leq \cdots \leq a_m$,
    and $n_j < n_{j+1}$ whenever $a_j = a_{j+1}$,
    with all $n_j < 0$.
    We define the grading on $\scrV_k (\frg)$ by
    \[
        \deg e^{a_1}_{n_1} \cdots e^{a_m}_{n_m} \ket{0}
        = -\sum_{j=1}^m n_j.
    \]
    We describe a vertex algebra structure on $\scrV_k (\frg)$ as follows.
    Write
    \[
        e^a (z) = \sum_{n} e^a_n z^{-n - 1}.
    \]
    In light of Theorem~\ref{thm-structure},
    we define
    \[
        e^a_{-n-1} (z) =
        \frac{1}{n!}
        \biggl( \frac {\partial} {\partial z} \biggr)^n e^a (z),
    \]
    and for a general element
    $e^{a_1}_{n_1} \cdots e^{a_m}_{n_m} \ket{0} \in \scrV_k (\frg)$,
    define
    \[
        \bigl( e^{a_1}_{n_1} \cdots e^{a_m}_{n_m} \ket{0} \bigr) (z)
        = \nord{ e^{a_1}_{n_1} (z) \cdots e^{a_m}_{n_m} (z) }
    \]
    to be the normally ordered product.
    The commutation relations can be written as the OPE
    \[
        e^a (z) \, e^b (w) =
        \frac {k \langle e^a, e^b \rangle} {(z - w)^2} +
        \frac {[e^a, e^b] (w)} {z - w} + \text{reg.}
    \]
    
    Finally, the $\bbZ$-grading on $\scrV_k (\frg)$
    is given by $\deg \ket{0} = 0$ and $\deg e^a_n = -n$.
    \varqed
\end{example}


\subsection{Modules}

\begin{definition}
    Let $\scrV$ be a vertex algebra.
    A \emph{$\scrV$-module} is a vector space $\scrM$,
    together with a linear map
    \begin{align*}
        Y_{\scrM} \colon \scrV & \to \End (\scrM) [[z, z^{-1}]], \\
        A & \mapsto A_{\scrM} (z) = \sum_{n \in \bbZ} A_{(n)}^{\scrM} (z),
    \end{align*}
    such that the following axioms hold:
    \begin{itemize}
        \item
            For all $A \in \scrV$ and $B \in \scrM$, we have
            \[
                A (z) \, B \in \scrM ((z)).
            \]
            In other words, we have
            \[
                A_{(n)} B = 0 \quad (n \gg 0).
            \]
        \item
            \textup{(Vacuum)}
            $\ket{0}_{\scrM} (z) = \operatorname{id}_{\scrM}$.
        \item
            \textup{(Associativity)}
            For any $A, B \in \scrV$ and $C \in \scrM$,
            the elements
            \begin{align*}
                A_{\scrM} (z) \, B_{\scrM} (w) \, C
                & \in \scrM ((z)) ((w)), \\
                (A_{\scrM} (z - w) \, B)_{\scrM} (w) \, C
                & \in \scrM ((w)) ((z - w))
            \end{align*}
            come from the same element of
            \[
                \scrM [[z, w]] [z^{-1}, w^{-1}, (z-w)^{-1}],
            \]
            in the sense analogous to that of Theorem~\ref{thm-assoc}.
    \end{itemize}
    If moreover, $\scrV$ is a $\bbZ$-graded vertex algebra,
    then $\scrM$ is called a \emph{graded $\scrV$-module},
    if it comes with a $\bbC$-grading
    \[
        \scrM = \bigoplus_{m \in \bbC} \scrM_m,
    \]
    such that
    \begin{itemize}
        \item
            For each $A \in \scrV_m$ and each $n \in \bbZ$,
            the operator $A_{(n)}^{\scrM}$ is homogeneous
            of degree $-n - 1 + m$.
    \end{itemize}
    If moreover, $\scrV$ is a conformal vertex algebra,
    then $\scrM$ is called a \emph{conformal $\scrV$-module}
    if it is a graded $\scrV$-module, and
    \begin{itemize}
        \item
            For all $m \in \bbC$,
            $\scrM_m$ is the $m$-eigenspace of the operator
            $L^{\scrM}_0 = L^{\scrM}_{(1)}$.
            \varqed
    \end{itemize}
\end{definition}

\begin{example}
    If $\scrV$ is a (plain, graded, or conformal) vertex algebra,
    then $\scrV$ itself is a (plain, graded, or conformal) $\scrV$-module.
    This is called the \emph{adjoint representation} of $\scrV$.
    \varqed
\end{example}

\begin{example}
    A subspace $\scrI \subset \scrV$ is called a \emph{vertex algebra ideal},
    if it is $T$-invariant,
    and satisfies $A (z) \, B \in \scrI ((z))$
    for all $A \in \scrV$ and $B \in \scrI$.
    Thus, the ideals of $\scrV$ are exactly
    the submodules of $\scrV$,
    regarded as a $\scrV$-module via the adjoint representation.
    \varqed
\end{example}


\subsection{Conformal blocks}

\begin{definition}
    Let $\scrF_n$ denote the space of meromorphic functions $f$ on $E^n$,
    where $E = \bbC / (\bbZ 1 \oplus \bbZ \tau)$ is an elliptic curve,
    such that $f$ is only allowed to have poles
    along the diagonals $z_i = z_j$ ($i \neq j$).
    
    Let $\scrV$ be a vertex algebra.
    An element of the \emph{conformal block} of $\scrV$
    on the elliptic curve $E$
    is a system $\{ \langle \cdots \rangle_n \}_{n \in \mathbb{N}}$
    of linear functions
    \begin{align*}
        \langle \cdots \rangle_n \colon \quad
        \scrV^{\otimes n} & \to \scrF_n, \\
        \scrO_1 \otimes \cdots \otimes \scrO_n & \mapsto
        \langle \scrO_1 (z_1) \cdots \scrO_n (z_n) \rangle,
    \end{align*}
    satisfying the following properties:
    \begin{itemize}
        \item
            (Symmetry)
            For each permutation $\sigma \in \mathfrak{S}_n$,
            \[
                \langle \scrO_1 (z_1) \cdots \scrO_n (z_n) \rangle =
                \pm \langle \scrO_{\sigma (1)} (z_{\sigma (1)}) \cdots
                \scrO_{\sigma (n)} (z_{\sigma (n)}) \rangle,
            \]
            where each $\scrO_i$ is either even or odd,
            and the $\pm$ sign is the sign of $\sigma$
            restricted on the odd elements.
        \item
            (Vacuum)
            \[
                \bigl\langle
                    \ket{0}(z) \,
                    \scrO_1 (z_1) \cdots \scrO_n (z_n)
                \bigr\rangle =
                \langle \scrO_1 (z_1) \cdots \scrO_n (z_n) \rangle.
            \]
        \item
            (Translation)
            \[
                \langle (T \scrO_1) (z_1) \cdots \scrO_n (z_n) \rangle =
                \frac{d}{d z_1}
                \langle \scrO_1 (z_1) \cdots \scrO_n (z_n) \rangle.
            \]
        \item
            (Residue)
            \[
                \Res_{z = z_1} \bigl(
                    \langle \scrO (z) \, \scrO_1 (z_1) \cdots \scrO_n (z_n) \rangle \,
                    (z - z_1)^k
                \bigr) =
                \langle \scrO_{(k)} \scrO_1 (z_1) \cdots \scrO_n (z_n) \rangle.
            \]
    \end{itemize}
    The \emph{conformal block} of $\scrV$ on $E$
    is the vector space of all such systems
    $\{ \langle \cdots \rangle_n \}$.
    \varqed
\end{definition}

\begin{remark}
    We are mainly concerned with genus $1$ conformal blocks
    in this paper, i.e.\ conformal blocks for elliptic curves,
    and not for curves of other genera.
    \varqed
\end{remark}

\begin{remark}
    Sometimes, we introduce a formal variable $\hbar$,
    and replace the space $\scrF_n$ by
    $\scrF_n [[\hbar, \hbar^{-1}]]$.
    We will consider this extended concept
    of a conformal block in section 4 below.
    \varqed
\end{remark}

Following Zhu \cite{zhu},
we sketch a way to construct elements of the conformal block
when the vertex algebra satisfies certain assumptions.
These elements are obtained as the trace in a module over the vertex algebra,
which is analogous to characters of associative algebras
being obtained as the trace in a representation of the associative algebra.

Zhu's construction uses a coordinate change
from the $\tau$-picture to the $q$-picture,
where $q = \upe^{2 \uppi \upi \tau}$.
In fact, note that the elliptic curve
\[
    E_\tau = \bbC / (z \sim z + 1 \sim z + \tau)
\]
is isomorphic to the elliptic curve
\[
    \bbC^\times / (w \sim w q),
\]
and this isomorphism is induced by the coordinate change
\[
    w = \upe^{2 \uppi \upi z}.
\]
For a given vertex algebra, we can rewrite everything in the new coordinate
\[
    \widetilde{z} = w - 1 = \upe^{2 \uppi \upi z} - 1,
\]
where the term $-1$ is added to ensure that $\widetilde{z}|_{z = 0} = 0$.
This defines a new vertex algebra structure on $\scrV$.

\begin{theorem}
    Let $(\scrV, Y, L, \ket{0})$ be a conformal vertex algebra,
    and define
    \begin{align*}
        \widetilde{Y} \colon \scrV &\to \End (\scrV) [[z, z^{-1}]], \\
        A &\mapsto A (\upe^{2 \uppi \upi z} - 1) \, \upe^{2 \uppi \upi z |A|},
    \end{align*}
    where $A \in \scrV$ is homogeneous.
    Then $(\scrV, \widetilde{Y}, \widetilde{L}, \ket{0})$
    is a conformal vertex algebra, where
    $\widetilde{L} = (2 \uppi \upi)^2 (L - c/24)$.
\end{theorem}

See \cite[Theorem~4.2.1]{zhu}.
The new vertex algebra structure is called the
\emph{transformed vertex algebra}.

\begin{theorem}[Zhu]
    \label{thm-zhu}
    Let $(\scrV, Y, L, \ket{0})$ be a conformal vertex algebra.
    Suppose that
    \begin{itemize}
        \item
            Every homogeneous space $\scrV_m$ is finite dimensional.
        \item
            $\scrV$ is spanned by elements of the form
            $L_{-i_1} \cdots L_{-i_n} a$,
            where $a \in \scrV$ satisfies $L_i a = 0$ $(i > 0)$,
            and $i_1, \dotsc, i_n > 0$.
        \item
            The quotient space $\scrV / C_2 (\scrV)$ is finite dimensional,
            where $C_2 (\scrV)$ is the subspace of $\scrV$
            spanned by the elements $a_{(-2)} b$ $(a, b \in \scrV)$.
    \end{itemize}
    Let $\scrM$ be a conformal $\scrV$-module, such that
    \begin{itemize}
        \item
            Every homogeneous space $\scrM_m$ is finite dimensional.
    \end{itemize}
    Then the $q$-trace
    \[
        \tr_{\scrM} \scrO_1 (z_1) \cdots \scrO_n (z_n) \, q^{L_0}
    \]
    converges absolutely in the domain
    $0 < |q| < |z_n| < \cdots < |z_1| < 1$,
    and has a meromorphic continuation on $0 < |q| < |z_i| < 1$
    $(i = 1, \dotsc, n)$,
    denoted by the same notation. Set
    \[
        \langle \scrO_1 (z_1) \cdots \scrO_n (z_n) \rangle _{\scrM} =
        \upe^{2 \uppi \upi z_1 |\scrO_1|} 
        \cdots \upe^{2 \uppi \upi z_n |\scrO_n|}
        \tr_{\scrM}
        \scrO_1 (\upe^{2 \uppi \upi z_1})
        \cdots \scrO_n (\upe^{2 \uppi \upi z_n}) \,
        q^{L_0} \, q^{-1/24}.
    \]
    Then $\langle \cdots \rangle _{\scrM}$
    is an element of the conformal block for the transformed vertex algebra
    $(\scrV, \widetilde{Y}, \widetilde{L}, \ket{0})$.
\end{theorem}

See \cite[Theorem~4.4.3]{zhu}.

\begin{remark}
    Theorem~\ref{thm-zhu} requires that
    $\scrV / C_2 (\scrV)$ is finite dimensional,
    which is quite a strong condition.
    In fact, none of the examples mentioned in \S\ref{sect-voa-examples}
    satisfy this condition. \varqed
\end{remark}
