\subsection{Definition}

Let $\scrV$ be a vertex algebra,
and let $I \in \scrV$ be an odd element such that $[I_{(0)}, I_{(0)}] = 0$.
One may consider the cohomology
\[
    H (\scrV, I_{(0)}).
\]
This space has a natural vertex algebra structure,
with the vertex operations and the translation operator
inherited from $\scrV$.
(See also \cite[\S5.7.3]{frenkel-ben-zvi}.)

\begin{definition}
    \label{def-brst-reduction}
    The vertex algebra $H (\scrV, I_{(0)})$
    is called the \emph{BRST reduction}
    of $\scrV$ with respect to $I$. \varqed
\end{definition}

We present two examples of this construction
in the following two subsections.


\subsection{Example: \texorpdfstring{\mathversion{normal}$\scrW$}{W}-algebras}

Throughout this subsection,
we fix a finite dimensional simple Lie algebra $\frg$ over $\bbC$,
with the triangular decomposition
\[
    \frg = \frn_- \oplus \frh \oplus \frn_+ .
\]
Let $\Delta_+$ be the set of positive roots of $\frg$,
and let $S = \{ \alpha_1, \dotsc, \alpha_{\ell} \}$
be the set of simple roots.

\begin{definition}
    Let $V$ be a vector space over $\bbC$,
    with a basis $\{ v_i \}_{i \in I}$.
    The \emph{Clifford vertex algebra} of $V$ is defined to be
    \[
        \Lambda_V = \bigotimes_{i \in I} \scrV^{\bc}_i,
    \]
    where each $\scrV^{\bc}_i$
    is a copy of the $\bc$ vertex algebra,
    defined in Example~\ref{eg-bc},
    and we write $b^i_n$ and $c_{i, n}$
    for the elements $b_n, c_n$ in $\scrV^{\bc}_i$.
    
    Apart from the natural $\bbZ$-grading on $\Lambda_V$,
    we define another $\bbZ$-grading, called the \emph{charge grading},
    by letting
    \[
        \operatorname{char} {}\ket{0} = 0, \quad
        \operatorname{char} b^i_{n} = -1, \quad
        \operatorname{char} c_{i, n} = 1.
    \]
    The parity on $\Lambda_V$ agrees with the charge grading modulo $2$.
    \varqed
\end{definition}

Let $k \in \bbC$, and consider the vertex algebra
\[
    C_k^\bullet (\frg) =
    \scrV_k (\frg) \otimes \Lambda_{\frn_+}^\bullet,
\]
where $\scrV_k (\frg)$ is the Kac--Moody vertex algebra of $\frg$
defined in Example~\ref{eg-kac-moody},
$\Lambda_{\frn_+}^\bullet$ is the Clifford vertex algebra
of $\frn_+$ defined above,
and the upper dot indicates the charge grading.

We define the \emph{standard differential}
\[
    d_{\mathrm{st}} \colon C_k^\bullet (\frg)
    \to C_k^{\bullet + 1} (\frg)
\]
as follows. Let
\[
    Q =
    \sum_{\alpha \in \Delta_+}
    e^{\alpha}_{-1} \ket{0} \otimes
    c_{\alpha, 0} \ket{0}
    - \frac{1}{2} \sum_{\alpha, \beta, \gamma \in \Delta_+}
    f^{\alpha \beta}_{\gamma} \ket{0} \otimes
    c_{\alpha, 0} \, c_{\beta, 0} \, b^{\gamma}_{-1} \ket{0},
\]
where $e^\alpha \in \frn_+$ is the basis element corresponding to $\alpha$,
the notation $\smash{e^\alpha_{-1}}$ is as in Example~\ref{eg-kac-moody},
and $\smash{f^{\alpha \beta}_{\gamma}}$
denotes the structure constants of $\frn_+$.
We have the corresponding vertex operator
\[
    Q (z) =
    \sum_{\alpha \in \Delta_+}
    e^{\alpha} (z) \otimes
    c_{\alpha} (z)
    - \frac{1}{2} \sum_{\alpha, \beta, \gamma \in \Delta_+}
    f^{\alpha \beta}_{\gamma} \bigl( 1 \otimes
    \nord {c_{\alpha} (z) \, c_{\beta} (z) \, b^{\gamma} (z)} \bigr),
\]
where the product in the first term is well-defined,
since $e^\alpha_n$ and $c_{\alpha, m}$ commute,
and each element $C_k^\bullet (\frg)$
is annihilated by all but finitely many terms
in the infinite sum arising from the product of the two power series.
We then set
\[
    d_{\mathrm{st}} = Q_{(0)}.
\]
One can verify that the OPE
$Q (z) \, Q (w)$ is regular, so that
\[
    d_{\mathrm{st}}^2 = \frac{1}{2} [ Q_{(0)}, Q_{(0)} ] = 0
\]
by Borcherd's identity.

We then twist the standard differential $d_{\mathrm{st}}$
by the element
\[
    \chi = \sum_{i=1}^\ell c_{\alpha_i, 0}
\]
which acts on $\Lambda^\bullet_{\frn_+}$,
given by the $0$-mode of the vertex operator
\[
    \chi (z) = \sum_{i=1}^\ell c_{\alpha_i} (z).
\]
Clearly, the OPE $\chi (z) \, \chi (w)$ is regular,
and one can verify that $Q (z) \, \chi (w)$ is also regular.
This means that
\[
    \bigl( C^\bullet_k (\frg), \ d = d_{\mathrm{st}} + \chi \bigr)
\]
is a chain complex,
often called the \emph{BRST complex}.

\begin{definition}
    The BRST reduction (Definition~\ref{def-brst-reduction})
    of the BRST complex is called the \emph{$\scrW$-algebra}, denoted by
    \[
        \scrW_k (\frg) = H^0 \bigl( C^\bullet_k (\frg), \ d \bigr),
    \]
    where $H^0$ is the only nonzero cohomology of the BRST complex,
    by Theorem~\ref{thm-w-alg} below. \varqed
\end{definition}

The structure of the $\scrW$-algebra
is well understood,
and is described as follows.

Recall that for a complex semi-simple Lie algebra $\frg$,
the Lie algebra cohomology
\[
    H^\bullet (\frg, \bbC)
\]
is freely generated by elements of degrees $2 d_i + 1$
($i = 1, \dotsc, r$, and $d_i \in \bbZ_{>0}$),
where $r$ is the rank of $\frg$, i.e.\ the dimension of the Cartan subalgebra.
The numbers $d_i$ are called the \emph{exponents} of $\frg$.
For example, the exponents of $\mathfrak{sl}_n$ are the integers $1, \dotsc, n - 1$.

\begin{theorem}
    \label{thm-w-alg}
    The $\scrW$-algebra $\scrW_k (\frg)$
    is generated by elements $W_i$ of degrees $d_i + 1$ $(i = 1, \dotsc, r)$,
    where the $d_i$ are the exponents of $\frg$.
    Precisely speaking, there exist elements
    \[
        W_{i,n} \in \scrW_k (\frg)
        \quad (i = 1, \dotsc, r; \enspace n < 0),
    \]
    such that the elements
    \[
        W_{i_1, n_1}^{m_1} \cdots W_{i_k, n_k}^{m_k} \ket{0},
    \]
    where $i_1 \leq \cdots \leq i_k$, and $n_j < n_{j+1}$ whenever $i_j = i_{j+1}$,
    with all $n_j < 0$ and $m_j > 0$,
    form a basis of $\scrW_k (\frg)$ as a vector space.
    
    Furthermore, $H^i \bigl( C^\bullet_k (\frg), \ d \bigr) = 0$ for $i \neq 0$.
\end{theorem}

See \cite[Theorem~15.1.9]{frenkel-ben-zvi}.


\subsection{Example: Poisson tensors}

In this subsection, we introduce an example of BRST reduction
that describes the Poisson $\sigma$-model,
as described in \cite[\S3.7]{voabv}.

Let $V \simeq \bbR^n$ be a vector space,
with a basis $\{ v_i \}_{i=1, \dotsc, n}$.
Consider the vertex algebra
\[
    \scrV =
    \bigotimes_{i \in I} {}
    (\scrV^{\betagamma}_i \otimes \scrV^{\bc}_i),
\]
where each $\scrV^{\betagamma}_i$ is
a copy of the $\betagamma$ vertex algebra,
and each $\scrV^{\bc}_i$
is a copy of the $\bc$ vertex algebra.
We write $\beta_i, \gamma^i, b^i, c_i$
for the corresponding vertex operators.

Write
\begin{align*}
    X^i (z) &= \gamma^i (z) + b^i (z) \, d z, \\
    \itTheta_i (z) &= c_i (z) + \beta_i (z) \, d z,
\end{align*}
where $X^i$ and $\itTheta_i$ are elements of
$\scrV \otimes \bbC [d z]$,
where $d z$ is a formal odd element,
so that $X^i$ is even and $\itTheta_i$ is odd.
We thus have the OPE
\[
    X^i (z) \, \itTheta_j (w) =
    \delta^i_j \, \frac{d z - d w}{z - w} + \text{reg.}
\]

Let
\[
    P = P^{ij} (x) \, \partial_i \wedge \partial_j
\]
be a Poisson tensor on $V \simeq \bbR^n$,
i.e.\ a bivector field satisfying $[P, P] = 0$,
where $[-, -]$ is the Schouten--Nijenhuis bracket.
Suppose that $P^{ij}$ are polynomial functions in $x$.
Then it makes sense to define
\[
    \widetilde{P} = P^{ij} (X) \, \itTheta_i \, \itTheta_j ,
\]
with its corresponding vertex operator
\[
    \widetilde{P} (z) =
    \nord{ P^{ij} (X) (z) \, \itTheta_i (z) \, \itTheta_j (z) },
\]
where the notation $P^{ij} (X) (z)$
only makes sense in a normally ordered product.

\begin{proposition}
    \label{prop-poisson}
    The OPE $\widetilde{P} (z) \, \widetilde{P} (w)$
    is regular at $z = w$.
\end{proposition}

\begin{proof}
    \allowdisplaybreaks
    The condition that $P$ is a Poisson tensor
    is equivalent to the coordinate expression
    \[
        (P^{kl} \, \partial_k P^{ij}) \,
        \partial_i \wedge \partial_j \wedge \partial_l = 0,
    \]
    as a $3$-vector field
    (i.e.\ degree $3$ polyvector field).
    
    On the other hand, 
    using Wick's theorem
    for the product of two normally ordered products
    \cite[Theorem~3.3]{kac},
    one computes the OPE
    \begin{align*}
        \widetilde{P} (z) \, \widetilde{P} (w)
        & = \nord{ P^{ij} (X) (z) \, \itTheta_i (z) \, \itTheta_j (z) } \ 
        \nord{ P^{kl} (X) (w) \, \itTheta_k (w) \, \itTheta_l (w) } \\
        & =
        \nord{ \widetilde{P} (z) \, \widetilde{P} (w) } \\
        & \quad {} + 2 \, \frac{d z - d w}{z - w} \Bigl(
        \nord{ \partial_k P^{ij} (X) (z) \, \itTheta_i (z) \, \itTheta_j (z) \,
        P^{kl} (X) (w) \, \itTheta_l (w) } \\
        & \quad \qquad {} -
        \nord{ P^{ij} (X) (z) \, \itTheta_j (z) \,
        \partial_i P^{kl} (X) (w) \, \itTheta_k (w) \, \itTheta_l (w) }
        \Bigr) \\
        & \quad {} + \cancel {\biggl( \frac{d z - d w}{z - w} \biggr)^2} \enspace
        ( \cdots ) \\
        & = 
        \nord{ \widetilde{P} (z) \, \widetilde{P} (w) },
    \end{align*}
    where we made use of the identity 
    $(P^{kl} \, \partial_k P^{ij}) \,
    \partial_i \wedge \partial_j \wedge \partial_l = 0$.
\end{proof}

Let $I (z)$ be the coefficient of $d z$ in $\widetilde{P} (z)$,
so that $I$ is an odd element of $\scrV$.
The above proposition immediately implies that the OPE
$I (z) \, I (w)$ is regular at $z = w$, so that
\[
    \Biggl[ \ 
        \oint I \ , \ \oint I
    \Biggr] = 0.
\]

The above argument can be summarized as follows.

\begin{theorem}
    For any Poisson tensor $P$ on $V \simeq \bbR^n$,
    let $I$ be the operator constructed above.
    Then the cohomology
    \[
        H (\scrV, I_{(0)})
    \]
    is a BRST reduction of $\scrV$,
    and carries a natural vertex algebra structure.
    \qed
\end{theorem}

% We introduce the twisting operator $Q$ by defining
% \[
%     Q (z) = \nord{ c_i (z) \, \partial_z \gamma^i (z) } \, d z,
% \]
% and letting $Q = Q_{(0)}$, so that
% \[
%     \begin{aligned}
%         Q b^i (z) & = \partial_z \gamma^i (z) \, d z, &
%         Q c_i (z) & = 0, \\
%         Q \beta_i (z) & = \partial_z c_i (z) \, d z, &
%         Q \gamma^i (z) & = 0.
%     \end{aligned}
% \]
% As a result, we have
% \begin{align*}
%     Q X^i (z) & = \partial X^i (z), \\
%     Q \itTheta_i (z) & = \partial \itTheta_i (z).
% \end{align*}

% \begin{proposition}
%     The OPE $Q (z) \, \widetilde{P} (w)$
%     is regular at $z = w$, so that
%     \[
%         [Q, \widetilde{P}_{(0)}] = 0.
%     \]
%     As a result, we have
%     \[
%         [P_{(0)} + Q, \ P_{(0)} + Q] = 0.
%     \]
% \end{proposition}

% \begin{proof}
%     ...
% \end{proof}

% Therefore, the cohomology
% \[
%     H (\scrV, P_{(0)} + Q)
% \]
% is a BRST reduction of $\scrV$.
