% \section{Introduction}
Motivated by (additive) categorifying Fomin--Zelevinsky's cluster algebra,
Buan--Marsh--Reineke--Reiten--Todorov \cite{BMRRT}
introduced cluster categories.
The theory of cluster categories has many links to other subjects, such as
\begin{itemize}
  \item commutative and non commutative algebra geometry,
    Calabi--Yau algebras, and stability conditions, cf.\ \cite{K5}, \cite{Q12};
  \item quiver representations and representation theory
    of finite-dimensional algebras,
    cf.\ \cite{BM}, \cite{BMRRT}, \cite{GLS}, \cite{IY2}.
\end{itemize}
Calabi--Yau algebras and Calabi--Yau categories play important roles
in the connection between cluster algebras and quiver representations.
For instance, Geiss--Leclerc--Schr\"{o}er \cite{GLS}
used the Calabi--Yau-2 property of the module category of preprojective algebras
to study certain cluster algebras associated to semisimple groups.
Iyama--Reiten \cite{IR} showed tilting modules for Calabi--Yau-2 algebras
and the connection with affine Weyl groups.
Particularly, they also studied the connection
between tilting theory on Calabi--Yau-3 algebras
and Fomin--Zelevinsky mutation of quivers.
Keller \cite{K3} constructed the cluster categories via orbit categories.
Amiot \cite{A} used generalized cluster categories
associated to some Calabi--Yau-3 dg algebras,
which satisfies the Calabi--Yau-2 property
and admits a canonical cluster-tilting object.
The generalized cluster category can be used
to categorify classes of cluster algebras \cite{K5}.

The article is motivated by the generalized cluster category
in Amiot--Guo--Keller's results \cite{A, Guo, K2}
and Ikeda--Qiu's results \cite[Theorem 6.7]{Q12}.
Let $\Bk$ be a field, $A$ be a homologically smooth
double differential grading $\Bk$-algebra and $G$ be a group.
The inverse dualizing complex $\Theta_A$
is defined by the cofibrant resolution of $\RHom_{A^e}(A, A^e)$.
Let $\theta \coloneq \Theta_A[g-1]$.
The \textit{Calabi--Yau-$g$ completion} of $A$ is defined as the tensor algebra
\[
  \Pi_g(A) \coloneq T_A(\theta)
  = A \oplus \theta \oplus (\theta \otimes_A \theta)\oplus \cdots,
\]
where $g \in G$.
Analogous to Keller's results \cite{K2},
the Calabi--Yau-$g$ completion $\Pi_gA$ of $A$ is homologically smooth,
and is a Calabi--Yau-$g$ algebra.
In particular, $\D_{fd}(\Pi_gA)$ is a Calabi--Yau-$g$ category.
Moreover, we notice that for an acyclic $\ZZ$-graded quiver $Q$,
the Calabi--Yau-$g$ completion of dg path algebra $\Bk Q$
is isomorphic to the Ginzburg $\ZZ \oplus G$-ddg algebra $\Gamma_g Q$ of $Q$.

Then consider the case when $G$ is a free abelian group $\ZZ \mathbb{X}$
of rank one generated by $\mathbb{X}$.
Suppose that $A$ is a basic finite-dimensional
dg $\Bk$-algebra of finite global dimension.
Denote $\per\Pi_{\XX}A$ (resp. $\D_{fd}(\Pi_{\XX}A)$)
the perfect derived category (resp. the finite dimensional derived category)
of $\Pi_{\XX}A$.
Then the generalized $\XX$-cluster category $\Ce(\Pi_{\XX}A)$
is defined by the quotient
\[
  \Ce(\Pi_{\XX}A) \coloneq \per \Pi_{\XX} A \quot \D_{fd}(\Pi_{\XX}A).
\]
We give the triangle equivalence
\begin{equation}\label{main}
  \per A \cong \Ce(\Pi_{\XX}A)
\end{equation}
between $\mathbb{X}$-cluster category and perfect derived category of $A$
in Theorem \ref{main thm}.
In Particular, if $A$ is the path algebra of some acyclic quiver $Q$,
we have the triangle equivalence \cite[Theorem 6.7]{Q1}
\[
  \per \Bk Q \xrightarrow{\cong} \Ce(\Gamma_{\XX}Q).
\]

If $A$ is obtained from a graded marked surface $\BS^{\lambda}$,
we have the triangle equivalence \cite[Theorem 4.7]{Q12}
\[
  \TFuk(\mathbf{S}^{\lambda}) \xrightarrow{\cong} \Ce_{\XX}(\TT).
\]
