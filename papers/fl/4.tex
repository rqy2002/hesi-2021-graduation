% \section{$\XX$-Cluster Category}
%=========================================================
%=========================================================
\subsection{Cluster category as Verdier quotient}
%=========================================================
Let $\Bk$ be a field and $A$ be a homologically smooth,
finite dimensional dg $\Bk$-algebra.
Replacing $g$ by an integer $N \geq 2$ in Definition \ref{CYcomp},
we obtain the usual Ginzburg dg algebra $\Pi_NA$
and the corresponding Calabi--Yau-$N$ category $\D_{fd}(\Pi_NA)$.
There is another way to realize cluster categories by Verdier quotient.
\begin{definition}
  With the assumption above, the \textit{generalized $N$-cluster category}
  is defined by the quotient as follow
  \[ \Ce(\Pi_NA) \coloneq \per \Pi_N A \quot \D_{fd}(\Pi_N A). \]
  In other words,
  it satisfies the short exact sequence of the triangulated categories:
  \[
    \begin{tikzcd}
      0 \ar[r] & \D_{fd}(\Pi_N A) \ar[r] & \per \Pi_N A \ar[r]
               & \Ce(\Pi_N A) \ar[r] & 0.
    \end{tikzcd}
  \]
\end{definition}

\begin{theorem}\cite{A, Guo, K2}
  Let $A$ be a finite-dimensional dg $\Bk$-algebra of finite global dimension.
  There is a natural triangle equivalence
  \[ \Ce_{N-1}(A) \cong \Ce(\Pi_N A) \]
  which matches their canonical cluster-tilting objects.
  In particular, $\Ce(\Pi_N A)$ is a Calabi--Yau-$(N-1)$ triangulated category.
\end{theorem}

\begin{remark}
  If $A$ is a homologically smooth Calabi--Yau-$3$ $\Bk$-algebra
  with non-positive degree, the category
  \[ \mathcal{C}_2(A) \coloneq \per A \quot \D_{fd}(A)\]
  is $\Hom$-finite and Calabi--Yau-$2$.
  The cluster-tilting object is the image of $A$
  in the quotient category $\mathcal{C}_2(A)$,
  whose endomorphism algebra is isomorphic to $H^0A$ \cite[Theorem 2.1]{A}.
  Generally, if $A$ is a Calabi--Yau-$N$ $\Bk$-algebra,
  where $N$ is an integer greater than $3$,
  the category
  \[ \mathcal{C}_N(A) \coloneq \per A \quot \D_{fd}(A)\]
  is $\Hom$-finite and Calabi--Yau-$(N-1)$,
  and the image of $A$ is an $(N-1)$-cluster-tilting object \cite[Theorem 2.2]{Guo}.
\end{remark}
We still replace $G$ by a free abelian group $\ZZ\XX$ of rank one
generated by $\XX$ in Definition \ref{CYcomp}.
We get the ddg algebra $\Pi_{\XX}A$
and the corresponding Calabi--Yau-$\XX$ category $\D_{fd}(\Pi_{\XX}A)$.
Consider the projection functor
\[ \pi_N \colon \Pi_{\XX}A \to \Pi_NA, \]
induced by the collapsing map
\[ \ZZ \oplus \ZZ \XX \ni (a, b) \mapsto a + bN \in \ZZ\]
We obtain functors of triangulated categories:
\[
  \pi_t \colon \D_{fd}(\Pi_{\XX}A) \to \D_{fd}(\Pi_NA)
  \quad \text{and} \quad
  \pi_N \colon \per\Pi_{\XX}A \to \per\Pi_NA.
\]
\begin{proposition}\cite{Q1}
  $\D_{fd}(\Pi_{\XX}A) \quot [\XX-N]$ (resp. $\per\Pi_{\XX}A \quot [\XX-N]$) is $N$-reductive.
  The triangulated structure is given by its unique triangulated hull
  $\D_{fd}(\Pi_NA) = \D_{fd}(\Pi_{\XX}A) \quott [\XX-N]$
  (reps. $\per\Pi_NA = \per\Pi_{\XX}A \quott [\XX-N]$).
\end{proposition}

%=========================================================
\subsection{Main theorem}
%=========================================================
\begin{definition}
  Let $\Bk$ be a field and $A$ be a homologically smooth dg $\Bk$-algebra.
  Generally, the \textit{$\XX$-cluster category} $\Ce(\Pi_{\XX}A)$
  is defined as
  \[
    \Ce(\Pi_{\XX}A)
    \coloneq \per \Pi_{\XX}A \quot \D_{fd}(\Pi_{\XX}A)
  \]
  via the Verdier quotient.
\end{definition}

\begin{theorem}\label{main thm}
  Let $A$ be a basic finite-dimensional dg $\Bk$-algebra of finite global dimension.
  The embedding $A \to \Pi_{\XX}A$ induces a triangle equivalence
  \[ \per A \xrightarrow{\cong} \Ce(\Pi_{\XX}A) \]
  which factors through $\per \Pi_{\XX}A$.
\end{theorem}
\begin{proof}
  Since there is an isomorphism
  $A \cong \bigoplus_{i}\widetilde{P_i}$ of dg $A$-modules,
  where $\widetilde{P_i}$'s are all projectives of $A$,
  and $\per A = \thick(A_A)$,
  $\per A$ is triangulated generated by its all projectives.
  $A$ is homologically smooth because the global dimension of $A$ is finite,
  which implies that the Calabi--Yau-$\XX$ completion $\Pi_{\XX}A$ of $A$
  is homologically smooth.
  The embedding $A \to \Pi_{\XX}A$
  naturally induces the functor between perfect derived category
  \[ i_*: \per A \to \per\Pi_{\XX}A \]
  which maps projectives to projectives.
  Let $\mathcal{C}$ denote the image of $i_*$,
  which is generated by projective dg $\Pi_{\XX}A$-modules
  $\{\Pi_{\XX}A[i]\}_{i \in \ZZ}$.

  By Theorem \ref{DfdCY},
  $\D_{fd}(\Pi_{\XX}A)$ is a Calabi--Yau-$\XX$ category,
  then the functor $\XX$ is a Serre functor in $\D_{fd}(\Pi_{\XX}A)$.
  Denote by $\Sim \Pi_{\XX}A$ the set of simple $\Pi_{\XX}A$-modules
  with both normal degree and $\XX$-degree $0$.
  Let $\mathcal{S}$ be the thick subcategory of $\D_{fd}(\Pi_{\XX}A)$
  generated by all objects in $\bigcup_{i \in \ZZ}\Sim \Pi_{\XX}A[i]$.
  For any dg module $M$ in $\D_{fd}(\Pi_{\XX}A)$,
  $\tau_{\leq 0}M$ is a dg $\Pi_{\XX}A$-submodule of $M$ and $\tau_{\leq 0}M$
  is the corresponding quotient dg $\Pi_{\XX}A$-module,
  and we have a distinguished triangle in $\D_{fd}(\Pi_{\XX}A)$
  \[ \tau_{\leq 0}M \to M \to \tau_{>0}M \to \tau_{\leq 0}M[1].\]
  Thus these two functors define a canonical unbounded t-structure
  \[ \D_{fd}(\Pi_{\XX}A) = \Pair{\mathcal{X}, \mathcal{Y}} \]
  on $\D_{fd}(\Pi_{\XX}A)$,
  where $\mathcal{X}$ is generated by $\bigcup_{j \geq 0} \mathcal{S}[j\XX]$
  and $\mathcal{Y}$ is generated by $\bigcup_{j < 0} \mathcal{S}[j\XX]$.
  Here, the notation $\Pair{\mathcal{A}, \mathcal{B}}$
  contains all object $M$ satisfying the following triangle
  \[ A \to M \to B \to A[1], \]
  where $A\in\mathcal{A}, B\in\mathcal{B}.$

  From the Calabi--Yau-$\XX$ duality, we deduce
  \[
    \Hom_{\per\Pi_{\XX}A}(M, \Pi_{\XX}A)
    \cong \DHom_{\per\Pi_{\XX}A}(\Pi_{\XX}A, M[\XX])
  \]
  for any $M \in \D_{fd}(\Pi_{\XX}A)$.
  Using the relation about $\Hom$ between projective objects and simple objects:
  \[
    \Hom^i_{\per\Pi_{\XX}A}(\Pi_{\XX}A[k\XX], \mathcal{S}[j\XX]) = 0
  \]
  for any $i \in \ZZ$ and $k \neq j$, we deduce
  \[
    \Hom^i_{\per\Pi_{\XX}A}(\mathcal{S}[j\XX],\Pi_{\XX}A[k\XX])
    \cong \DHom^i_{\per\Pi_{\XX}A}(\Pi_{\XX}A[k\XX], \mathcal{S}[(j+1)\XX]) = 0
  \]
  for any $i \in \ZZ$ and $k \neq j+1$. Therefore, we have
  \begin{equation}\label{eq1}
    \Hom_{\per\Pi_{\XX}A}(\mathcal{S}[j\XX], \mathcal{C}[k\XX]) = 0
  \end{equation}
  for $k \neq j+1$ in each degree. Moreover, we have
  \begin{align*}
    \Hom_{\per\Pi_{\XX}A}(\Pi_{\XX}A[j\XX], \mathcal{S}[j\XX])
    &\cong \DHom_{\per \Pi_{\XX}A}(\mathcal{S}[(j-1)\XX], \Pi_{\XX}A[j\XX]) \\
    &\cong \DHom_{\per \Pi_{\XX}A}(\mathcal{S}[j\XX], \Pi_{\XX}A[(j+1)\XX]),
  \end{align*}
  where the Hom-space
  \[ \Hom_{\per\Pi_{\XX}A}(\Pi_{\XX}A[j\XX], \mathcal{S}[j\XX]) \neq 0. \]
  Hence we have
  \begin{equation}\label{eq2}
    \Hom(\mathcal{S}[j\XX], \mathcal{C}[(j+1)\XX]) \neq 0.
  \end{equation}
  The right perpendicular of $\mathcal{X}$ in $\per\Pi_{\XX}A$ is defined as
  \[
    \mathcal{X}^{\bot} \coloneq
    \{Z \in \per\Pi_{\XX}A \mid \Hom_{\per\Pi_{\XX}A}(\mathcal{X},Z) = 0\}.
  \]
  From \eqref{eq1} and \eqref{eq2},
  we deduce that $\mathcal{X}^{\bot}$ is generated by
  $\bigcup_{j \leq 0}\mathcal{C}[j\XX].$

  On the other hand, since
  \[
    \Hom^i_{\per\Pi_{\XX}A}(\Pi_{\XX}A[k\XX], \mathcal{S}[j\XX])
    = \delta_{j,k}\delta_{i,0},
  \]
  we obtain
  \begin{equation}\label{eq3}
    \Hom_{\per\Pi_{\XX}A}(\mathcal{C}[k\XX], \mathcal{S}[j\XX]) = 0
  \end{equation}
  for $k \neq j$ in each degree and
  \begin{equation}\label{eq4}
    \Hom_{\per\Pi_{\XX}A}(\mathcal{C}[j\XX], \mathcal{S}[j\XX]) \neq 0.
  \end{equation}
  By definition of $\mathcal{Y}$, we have $\mathcal{Y}[1] = \mathcal{Y}$.
  The left perpendicular of $\mathcal{Y}$ in $\per\Pi_{\XX}A$ is defined as
  \[
    {}^{\bot}\mathcal{Y} \coloneq
    \{Z \in \per\Pi_{\XX}A \mid \Hom_{\per\Pi_{\XX}A}(Z,\mathcal{Y}) = 0\}.
  \]
  From \eqref{eq3} and \eqref{eq4}, we deduce that $^{\bot}\mathcal{Y}[1]$,
  which equals to $^{\bot}\mathcal{Y}$,
  is generated by $\bigcup_{j \geq 0}\mathcal{C}[j\XX].$
  Therefore, we have
  \[ \mathcal{X}^{\bot} \cap {}^{\bot}\mathcal{Y}[1] = \mathcal{C}. \]
  In order to use Theorem 1.1 in \cite{IY},
  we need to prove the following proposition.
  \phantom{\qedhere}
\end{proof}

\begin{proposition}
  Both $\Pair{\mathcal{X}, \mathcal{X}^{\bot}}$
  and $\Pair{{}^{\bot}\mathcal{Y}, \mathcal{Y}}$
  are torsion pairs in $\per\Pi_{\XX}A$.
\end{proposition}
\begin{proof}
  From the discussion above,
  it suffices to show that they generate $\per\Pi_{\XX}A$.
  Since $A$ is quasi-isomorphic to the dg path algebra $\Bk Q$
  of some graded quiver $Q$ with relation $R$.
  By adding new arrows and differential to $Q$ in Oppermann's construction \ref{Opp},
  we may assume that $A = \Bk Q$ where $Q$ is a graded quiver.
  By Theorem \ref{3.10} we have $\Pi_{\XX} A = \Pi_{\XX} \Bk Q = \Gamma_{\XX}Q$
  where $\Gamma_{\XX}Q$ is the Ginzburg algebra of quiver $Q$. Before proving
  \[
    \per\Gamma_{\XX}Q = \Pair{{}^\bot\mathcal{Y}, \mathcal{Y}}, \tag{i}
  \]
  we claim that: $\mathcal{C} = \Pair{\Ce[\XX], \mathcal{S}}$.
  We first prove the claim.
  Take any generator, say projective $P_i \in \mathcal{C}$ for $i \in Q_0$.
  Consider the standard triangle
  \begin{equation}\label{5.1}
    S_i[-1] \to A_1 \oplus B_1 \to P_i \to S_i
  \end{equation}
  where
  \[
    A_1 = \bigoplus_{i \xrightarrow{a} j \in Q_1} P_j
    \quad \text{and} \quad
    B_1 = \bigoplus_{i \xrightarrow{b} k \in Q_1^*\cup Q_0^*} P_k[\XX],
  \]
  because there is no differential between $A_1$ and $B_1$.
  Since the $\XX$-degree of arrows $b$'s $\in Q_1^*\cup Q_0^*$ are $-1$,
  we have $A_1 \in \mathcal{C} $ and $B_1 \in \Ce[\XX]$.
  We decompose $P_j$'s by (\ref{5.1}) and get the following filtration of $P_i$:
  \[
    \begin{tikzcd}
      A_2 \oplus B_2 \ar[rr]
      &[-15pt] &[-15pt] A_1 \oplus B_1 \ar[rr] \ar[dl]
      & & P_i \ar[dl] \\
      & \bigoplus\limits_{i \xrightarrow{a} j \in Q_1} S_j \ar[ul, dashed]
      & & S_i \ar[ul, dashed]
    \end{tikzcd}
  \]
  where
  \[
    A_2 = \bigoplus_{\substack{
      i \xrightarrow{a} j \in Q_1 \\ j \xrightarrow{a'} j' \in Q_1
    }} P_{j'}
    \quad \text{and} \quad
    B_2 = B_1 \oplus \bigoplus_{\substack{
      i \xrightarrow{a} j \in Q_1 \\ j \xrightarrow{b'} k' \in Q_1^*\cup Q_0^*
    }}P_{k'}[\XX].
  \]
  As before, we have $A_2 \in \mathcal{C} $ and $B_2 \in \Ce[\XX]$.
  Then decompose $P_{j'}$'s and repeat the process above.
  Since $\Bk Q$ is finite dimensional,
  this filtration will end up in finite length, i.e.,
  \begin{equation}\label{5.2}
    \begin{tikzcd}[column sep=.9pc]
      B_{l + 1} \ar[rr]
      & & A_l \oplus B_l \ar[rr] \ar[dl]
      & & \cdots\cdots \ar[r] \ar[dl]
      & A_1 \oplus B_1 \ar[rr]
      & &[5pt] P_i\ar[dl] \\
      & H_{l+1} \ar[ul, dashed]
      & & H_l \ar[ul, dashed]
      & & & H_1\ar[ul, dashed]
    \end{tikzcd}
  \end{equation}
  where
  \[
    H_m = \bigoplus_{\substack{
      i \stackrel{p}{\to \cdots \to} t(p) \\ \|p\|_{Q_1} = m-1
    }} S_{t(p)},
    \quad A_m = \bigoplus_{\substack{
      i \stackrel{p}{\to \cdots \to} t(p) \\ \|p\|_{Q_1} = m
    }}P_{t(p)}
    , \quad \text{and}\ 
    B_m = \bigoplus_{\substack{
      i \stackrel{p}{\to \cdots \to} t(p) \xrightarrow{b} t(b)
      \\ \|p\|_{Q_1} < m, b \in Q_1^*
    }}P_{t(b)}[\XX].
  \]
  Here, the notation $\|p\|_{Q_1}$ represents the length of $p$ in $Q_1$.
  As above, we have $B_m \in \Ce[\XX]$ and $H_m \in \mathcal{S}$.
  By (\ref{5.2}), we have
  \[ P_i \in \Pair{\Ce[\XX], \mathcal{S}} \]
  for $\forall i \in Q_0$.

  Since $\Ce$ is generated by $\{P_i\}_{i \in Q_0}$,
  it follows that
  \[ \Ce \subseteq \Pair{\Ce[\XX], \mathcal{S}}. \]
  Similarly, we have
  \[\Ce[-m\XX] \subseteq \Pair{\Ce[-m\XX+\XX], \mathcal{S}[-m\XX]} \]
  for any positive integer $m$.
  By induction, we have
  \[
    \bigcup_{j<0} \Ce[\XX]
    \subseteq \BigPair{\Ce, \bigcup_{j<0} \mathcal{S}[\XX]}.
  \]
  Therefore, we have
  \[
    \per\Gamma_{\XX}Q
    = \BigPair{\bigcup_{j \geq 0}\Ce[\XX], \bigcup_{j<0}\mathcal{S}[\XX]}
    = \Pair{{}^\bot\mathcal{Y}, \mathcal{Y}}.
  \]
  Before proving the other statement
  \[
    \per\Gamma_{\XX}Q = \Pair{\mathcal{X}, \mathcal{X}^\bot}, \tag{ii}
  \]
  we also make the claim that $\mathcal{C}[\XX] = \Pair{\mathcal{S},\Ce}$.
  We then prove the claim.
  As above, we first take any generator,
  say projective $P_i \in \mathcal{C}$ for $i \in Q_0$
  and rewrite (\ref{5.1}) as:
  \begin{equation}\label{5.3}
    S_i \to A_1[1] \oplus B_1[1] \to P_i[1] \to S_i[1].
  \end{equation}
  Rewrite $A_1[1] \oplus B_1[1]$ as
  $A_1[1] \oplus B_1[1] = P_i[\XX] \oplus A_1[1] \oplus \underline{B_1}[1]$
  where $\underline{B_1} = \bigoplus_{i\xrightarrow{b} k \in Q_1^*}P_k[\XX]$.
  Then we have three triangles:
  \begin{itemize}
    \item $S_i \to A_1[1] \oplus B_1[1] \to P_i[1] \to S_i[1]$.
    \item $A_1 \oplus \underline{B_1} \to P_i[\XX]
      \to A_1[1] \oplus B_1[1] \to A_1[1]\oplus \underline{B_1}[1]$.
    \item Since there exists morphism $S_i \to P_i[\XX]$
      from Calabi--Yau-$\XX$ duality,
      it can be completed to a triangle $S_i \to P_i[\XX] \to M \to S_i[1]$.
  \end{itemize}
  By the Octahedron Axiom, the third column is a triangle
  \begin{equation}\label{5.4}
    \begin{tikzcd}
      & A_1 \oplus \underline{B_1} \ar[r, equal] \ar[d]
      & A_1 \oplus \underline{B_1}\ar[d, dashed] \\
      S_i\ar[d, equal] \ar[r] & P_i[\XX]\ar[d] \ar[r]
      & M\ar[d, dashed] \ar[r] & S_i[1] \ar[d, equal] \\
      S_i \ar[r] & A_1[1] \oplus B_1[1] \ar[d] \ar[r]
      & P_i[1] \ar[d, dashed] \ar[r] & S_i[1] \\
      & A_1[1] \oplus \underline{B_1}[1]\ar[r, equal]
      & A_1[1]\oplus \underline{B_1}[1].
    \end{tikzcd}
  \end{equation}
  Hence we have
  $P_i[\XX] \in \Pair{S_i, A_1 \oplus \underline{B_1} \oplus P_i[1]}$
  where $S_i \in \mathcal{S}, A_1, P_i[1] \in \Ce$
  and $\underline{B_1} \in \Ce[\XX]$.
  Using (\ref{5.4}) for $P_k[\XX]$'s in $\underline{B_1}$,
  we get the following filtration of $P_i[\XX]$:
  \[
    \begin{tikzcd}[column sep=.5pc]
      &|[xshift=3.5pc]| S_i \ar[rr] & & P_i[\XX] \ar[dl] \\
      \bigoplus_{i \xrightarrow{b} k \in Q_1^*} S_k \ar[rr]
      & & A_1 \oplus \underline{B_1} \oplus P_i[1] \ar[ul, dashed] \ar[dl] \\
      & A_2 \oplus \underline{B_2} \oplus P_k[1] \oplus P_i[1] \ar[ul, dashed].
    \end{tikzcd}
  \]
  where
  \[
    A_2 = A_1 \oplus \bigoplus_{\substack{
      i \xrightarrow{b} k \in Q_1^* \\ k \to j' \in Q_1
    }}P_{j'}
    \quad \text{and} \quad
    \underline{B_2} = \bigoplus_{\substack{
      i \xrightarrow{b} k \in Q_1^* \\ k \to k' \in Q_1^*
    }}P_{k'}[\XX].
  \]
  As above, we have $A_2 \in \Ce, \underline{B_2} \in \Ce[\XX]$.
  Then decompose $P_{k'}[\XX]$'s and repeat the process.
  Since $\Bk Q$ is finite dimensional,
  this filtration will end up in finite step as:
  \begin{equation}\label{5.5}
    \begin{tikzcd}[column sep=-5pt, row sep=.9pc, scale cd=0.8]
      P_i[\XX]\ar[rr]
      &[10pt] & A_1 \oplus \underline{B_1} \oplus C_1 \ar[rr] \ar[dl, dashed]
      & & A_2\oplus \underline{B_2} \oplus C_2 \ar[r] \ar[dl, dashed]
      &[10pt] \cdots \ar[rr]
      &[15pt] & A_l\oplus \underline{B_l} \oplus C_l \ar[rr]\ar[dl, dashed]
      & & A_{l+1} \oplus C_{l+1} \ar[dl, dashed] \\
      & H_1 \ar[ul] & & H_2 \ar[ul] & & & H_l \ar[ul] & & H_{l+1} \ar[ul]
    \end{tikzcd}
  \end{equation}
  where
  \begin{align*}
    H_m &= \bigoplus_{\substack{
      i \stackrel{b}{\to \cdots \to} t(b) \\ \|b\|_{Q_1^*} = m-1
    }}S_{t(b)},
    & A_m &= \bigoplus_{\substack{
        i \stackrel{b}{\to \cdots \to} t(b) \xrightarrow {p} t(p) \\
        \|b\|_{Q_1^*} < m, p \in Q_1
      }}P_{t(p)},\\
    \underline{B_m} &= \bigoplus_{\substack{
      i \stackrel{b}{\to \cdots \to} t(b) \\ \|b\|_{Q_1^*} = m
    }} P_{t(b)}[\XX],
    & C_m &= \bigoplus_{\substack{
      i\stackrel{b}{\to \cdots \to} t(b) \\ \|b\|_{Q_1^*} = m
    }} P_{t(b)}[1].
  \end{align*}
  Here, the notation $\|p\|_{Q_1^*}$ represents the length of $p$ in $Q_1^*$.
  As above, we have $A_m, C_m \in \Ce$ and $H_m \in \mathcal{S}$.
  By (\ref{5.5}), we have
  \[ P_i[\XX] \in \Pair{\mathcal{S}, \Ce}\]
  for $\forall i \in Q_0$.

  Since $\Ce$ is generated by $\{P_i\}_{i \in Q_0}$, it follows that
  \[ \Ce[\XX] \subseteq \Pair{\mathcal{S}, \Ce}. \]
  Similarly, we have
  \[ \Ce[m\XX] \subseteq \Pair{\mathcal{S}[(m-1)\XX], \Ce[(m-1)\XX]} \]
  for any positive integer $m$.
  By induction, we have
  \[
    \bigcup_{j>0} \Ce[\XX]
    \subseteq \BigPair{\bigcup_{j \geq 0} \mathcal{S}[\XX], \Ce}.
  \]
  Therefore, we have
  \[
    \per\Gamma_{\XX}Q
    = \BigPair{\bigcup_{j \geq 0}\mathcal{S}[\XX], \bigcup_{j \leq 0}\Ce[\XX]}
    = \Pair{\mathcal{X}, \mathcal{X}^\bot}. \qedhere
  \]
\end{proof}

\begin{proof}[Proof of Theorem \ref{main thm}]
  By \cite[Theorem 1.1]{IY}, the composition
  \[
    \mathcal{X}^{\bot} \cap ^{\bot}\mathcal{Y}[1]
    = \mathcal{C} \hookrightarrow \per\Pi_{\XX}A
      \twoheadrightarrow \mathcal{C}(\Pi_{\XX}A)
    = \per \Pi_{\XX}A \quot \D_{fd}(\Pi_{\XX}A)
  \]
  induces an equivalence $F$
  \[
    \begin{tikzcd}
      \mathcal{C} \ar[dr] \ar[rr, "{\cong, F}"]
      & & \mathcal{C}(\Pi_{\XX}A) \\
      &\per\Pi_{\XX} A\ar[ur]
    \end{tikzcd}
  \]
  between additive categories.
  In particular,
  $\mathcal{C}$ has a structure of a triangulated category induced by $F$.
  Finally, since $i_*$ is a triangulated functor, the functor
  \[ F\circ i_*: \per A \to \mathcal{C}(\Pi_{\XX}A) \]
  is an equivalence as triangulated categories.
\end{proof}

\begin{remark}
  Suppose $A$ is a dg $\Bk$-algebra with non-positive degree.
  There exists a canonical correspondence
  between the silting object $A$ in $\per A$
  and the cluster-tilting object in $\mathcal{C}(\Pi_{\XX}A)$.
  More precisely, the image of $A$ under the equivalence
  is a cluster-tilting object
  in the $\XX$-cluster category $\mathcal{C}(\Pi_{\XX}A)$.
\end{remark}

\subsection{Special cases}
Theorem \ref{main thm} can be applied into these two following special cases.

\begin{example}\cite{Q1}
  Suppose $A$ is a path algebra of some acyclic quiver $Q$, namely $A =\Bk Q$.
  By Theorem \ref{main thm} above,
  we have a canonical triangle equivalence in \cite[Theorem 6.7]{Q1}
  \[ \per \Bk Q \xrightarrow{\cong} \Ce(\Gamma_{\XX}Q). \]
\end{example}

\begin{example}\cite{Q12}
  Let $\mathbf{S}^{\lambda} = (\mathbf{S},\mathbf{M},\mathbf{Y},\lambda)$
  be a graded marked surface,
  where $\lambda$ is the grading on $\mathbf{S}$
  and $\mathbf{T}$ is its open arc system.
  Then we consider the graded quiver with relation $(Q^{(0)}_\TT,R^{(0)}_\TT)$
  associated to $\mathbf{S}^{\lambda}$ where
  \begin{itemize}
    \item the vertices of $Q^{(0)}_\TT$ are the open arcs in $\mathbf{T}$;
    \item the arrows of $Q^{(0)}_{\mathbf{T}}$
      are the anticlockwise angles of open arcs in $\mathbf{T}$
      and the grading of each arrow is given by the intersubsection;
    \item the quadratic relations of $R^{(0)}_\TT$ are composable arrows
      whose corresponding angles do not share a marked point.
  \end{itemize}
  By Oppermann's construction \ref{Opp},
  we can add some new arrows and differential
  to kill the relations in $R^{(0)}_\TT$ to get a graded quiver $(Q_{\TT}, d)$
  without relation such that $\mathcal{A}_{\TT} \coloneq \Bk Q_{\TT}$
  is quasi-isomorphic to $\Bk Q^{(0)}_{\TT}$.
  By Calabi--Yau-$\XX$ completion of $\Bk Q_{\TT}$,
  we obtain the Ginzburg $\ZZ\oplus \ZZ\XX$-dg algebra,
  denoted by $\Gamma_{\TT}^{\XX}$.
  By Theorem \ref{main thm} above,
  we have a canonical triangle equivalence in \cite[Theorem 4.7]{Q12}
  \[ \TFuk(\mathbf{S}^{\lambda}) \xrightarrow{\cong} \Ce_{\XX}(\TT), \]
  where $\TFuk(\mathbf{S}^{\lambda}) \coloneq \per \mathcal{A}_{\TT}$
  denotes the topological Fukaya category and
  $\Ce_{\XX}(\TT) \coloneq \per \Gamma_{\TT}^{\XX} \quot \D_{fd}(\Gamma_{\TT}^{\XX})$
  represents the $\XX$-cluster category.
\end{example}
%=========================================================
\subsection{Future work}
%=========================================================
\begin{corollary}\label{Cor}
  With the assumption in Theorem \ref{main thm},
  we have the following commutative diagram of triangulated categories:
  \[
    \begin{tikzcd}
      0 \ar[r] & \D_{fd}(\Pi_{\XX}A) \ar[r] \ar[d, "{\quott [\XX-N]}"]
               & \per\Pi_{\XX}A \ar[r] \ar[d, "{\quott [\XX-N]}"]
               & \per A \ar[r] \ar[d, "{\tau\circ[2-N]}"] & 0 \\
      0\ar[r] & \D_{fd}(\Pi_NA) \ar[r] & \per\Pi_NA \ar[r] & \Ce_{N-1}(A) \ar[r] &0,
    \end{tikzcd}
  \]
  where both lines are short exact sequences.
\end{corollary}

\begin{proof}
  Theorem \ref{main thm} implies the exactness of the first row
  and Amiot-Guo-Keller's results in \cite{A,Guo,K2}
  imply the exactness of the second row.
  The left square commutes
  because both the second horizontal functors are natural inclusions
  and the first two vertical functors are natural $N$-reductive projections.
  The right square commutes
  because the generators (i.e., projectives) of $\per \Pi_{\XX}A$
  are mapped to cluster-tilting objects
  in $\mathcal{C}_{N-1}(A)$ in both two ways.
\end{proof}

\begin{definition}
  Let $\mathcal{T}$ be a triangulated category.
  A \textit{dg enhancement} of $\mathcal{T}$
  is a pretriangulated dg category $\mathcal{A}$
  together with a triangle equivalence
  $F \colon \mathcal{T} \to H^0(\mathcal{A})$.
\end{definition}

\begin{remark}
  A triangulated category $\mathcal{T}$ has a dg enhancement
  if and only if it is algebraic.
  In particular, the derived category of some dg algebra $A$ has a dg enhancement.
\end{remark}

\begin{conjecture}
  In general,
  suppose that a triangulated category $\mathcal{T}$ has a dg enhancement
  and $F$ is an autoequivalence of $\mathcal{T}$,
  then there exists a unique triangulated hull of $\mathcal{T} \quot F$.
\end{conjecture}

In our case, since both the perfect derived category $\per\Pi_{\XX}A$
and the finite dimensional category $\D_{fd}(\Pi_{\XX}A)$
are full subcategories of the derived category of $\mod A$,
which has a dg enhancement, thus they have dg enhancements
and so does their Verdier quotient
\[ \mathcal{C}(\Pi_{\XX}A) \cong \per A. \]
Since the three vertical arrows in Corollary \ref{Cor} are all auto equivalences,
the diagram of short exact sequences naturally commutes.
