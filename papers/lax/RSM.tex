\subsection{Basic definitions}

Now we introduce the theory of symmetric manifolds (or called
symmetric spaces). Firstly we give some basic geometric
properties, learning more about $I(M)$ in passing.

\begin{definition}
Let $M$ be a Riemannian manifold. $M$ is a  Riemannian (globally)
symmetric
if each $p \in M$ is an isolated fixed point of an involutive
isometry $s_{p}$ of $M$.	
\end{definition}
\begin{definition}
	A mapping is called involutive if its square, but not the
	mapping itself, is
	the identity.
\end{definition}
For an isometry, it has the following properties:
\begin{lemma}\label{31}
	Let $\left( M, g_{M} \right)$ and $\left( N, g_{N} \right)$
	be two Riemannian manifolds and $F\: M \rightarrow N$ be a
	local Riemannian isometry. Then
	
  \begin{enumerate}
    \item $F$ maps geodesics to geodesics.
    
    \item For any $p \in M$ and $v \in T_{p} M$, if $\exp _{p} v$
    is defined, then $\exp _{F(p)}\left( D F_{p}(v) \right)$ is
    defined and $F \circ \exp _{p}(v) = \exp _{F(p)} \circ D
    F_{p}(v) .$ \qedhere
  \end{enumerate}
\end{lemma}
\bproof \hfill
\begin{enumerate}
  \item is an obvious conclusion in Riemannian geometry.

  \item If $\exp _{p}(v)$ is defined, then $t \mapsto \exp _{p}(t v)$
  is a geodesic. Thus $t \mapsto$ $F\left( \exp _{p}(t v) \right)$
  is also a geodesic. Since
  \[
  \begin{aligned}
    \left.\frac{d}{d t} F\left( \exp _{p}
    \right)\right|_{t=0} &=D F\left( \left .\frac{d}{d t} \exp
    _{p}(t v) \right|_{t=0} \right) =D F(v)
  \end{aligned}
  \]
  we have that $F\left( \exp _{p}(t v) \right)=\exp _{F(p)}(t D
  F(v))$. Setting $t=1$ then the claim is proved.
\end{enumerate}
\eproof
\begin{corollary}\label{11.2}
Let $F, G\:\left( M, g_{M} \right) \rightarrow \left( N, g_{N}
\right)$ be two local Riemannian isometries. If $M$ is connected
and there exists $p \in M$ such that $F(p) = G(p), D F_{p} = D
G_{p}$, then
$F=G$ on $M$.	
\end{corollary}
\bproof
Let
\[
A=\left\{ x \in M \mid F(x) = G(x), D F_{x} = D G_{x} \right\}
\]
We know that $p \in A$ and that $A$ is closed. Property (2) of
Lemma $\ref{31}$ tells
that
\[
\begin{aligned}
	F \circ \exp _{x}(v) & = \exp _{F(x)} \circ D F_{x}(v) =\exp
	_{G(x)} \circ D G_{x}(v) = G \circ \exp _{x}(v)
\end{aligned}
\]
if $x \in A$. Since $\exp _{x}$ maps onto a neighborhood of $x$
it follows that some
neighborhood of $x$ also lies in $A$. This shows that $A$ is open
and hence 
$A=M$ by connectedness.
\eproof

\begin{theorem}
	Let $M$ be a Riemannian globally symmetric manifold. Then $M$
	is geodesically complete and for any $p \in M$ and any
	involutive isometry $s_{p} \in I(M)$ with $p$ as an isolated
	fixed point, it holds that $s_{p}\left( \exp _{p} v
	\right)=\exp _{p}(-v), \forall v \in T_{p} M$.
	Moreover, $I(M)$ is transitive on $M$.
\end{theorem}
\bproof
Let $A=\left( d s_{p} \right)_{p}$. Then $A$ is a linear
transformation on $T_{p} M$ and $A^{2}=I$. So $T_{p} M$ can be
decomposed into $T_{p} M = V^{+} \oplus V^{-}$, where $V^{+}$ is
the eigenspace of 1 and $V^{-}$ is the eigenspace of $-1 .$
Suppose $V^{+} \neq 0$ and
$X \neq 0 \in V^{+}$. Let $\gamma$ be a geodesic starting at $p$
with tangent vector $X$ at $p$.
Then $s_{p}(\gamma)$ is also a geodesic due to Lemma \ref{31} and
with tangent vector $X$ at $p$. So $\gamma$ coincides with
$s_{p}(\gamma)$, hence $p$ is not an isolated fixed point of
$s_{p}$,
a contradiction. So $A=-I .$

Then for any geodesic $\gamma = \exp _{q}(t v), 0 \leq t \leq 1,
v \in T_{q} M$, let $p = \exp _{q}(v) .$ By
Lemma $\ref{31}$, $s_{p}(\gamma)$ is also a geodesic and it
extends $\gamma$ to $0 \leq t \leq 2$. Repeat the process, we
know that $\gamma$ can be extended to $\gamma\: \mathbb{R}
\rightarrow M .$ So $M$ is geodesically complete and $\exp _{p}$
can be defined on $T_{p} M$ for any $p \in M$. So by Lemma
$\ref{31}$ , for any $p \in M$ and $v \in T_{p} M$, we have
$s_{p}\left( \exp _{p}(v) \right)=\exp _{p}(A v)=\exp _{p}(-v)$.
Finally, by Hopf-Rinow theorem, any two points $p, q$ can be
joined by a geodesic $\gamma\:[0,1] \rightarrow M$ such that
$\gamma(0) = p, \gamma(1) = q$. Let $m=\gamma\left( \frac{1}{2}
\right)$, then $s_{m}(p)=q, s_{m}(q)=p .$ So $I(M)$ acts
transitively on $M$.
\eproof

	By now we've proved that $I(M)$ is a locally compact
	transitive transformation group on $M$. Also from above we
	see that the involutive isometry at each point is unique. And
	$M$ is homeomorphic to $I(M) / I(M)_p$. To see the
	smoothness, we need a little more work as follows.


\begin{lemma}\label{3.7}
Let $M$ be a Riemannian globally symmetric manifold and $\gamma\:
\mathbb{R} \rightarrow M$ be a geodesic with $\gamma(0) = p_{0}
.$ Then for any $t \in \mathbb{R}$, the map $T_{t} \coloneq
s_{\gamma\left( \frac{t}{2} \right)} s_{p_{0}}$ is an isometry of
$M$, called a transvection, which sends $p_{0}$ to $\gamma(t)$
and $\left( d T_{t} \right)_{p_{0}}$ is the parallel translation
from $p_{0}$ to $p_{t} = \gamma(t)$.

Moreover, we have $T_t(\ga(s)) = \ga(t+s)$ and $T_t\circ T_s =
T_{t+s}$, hence $\{ T_t \}$ is a one-parameter group of
isometries on $M$.	
\end{lemma}
\bproof
Denote the point $\gamma(t)$ by $p_{t}$ and $s_{\gamma(t)}$ by
$s_{t}, \forall t \in \mathbb{R} .$ Let $\tau_{t}$ and
$\tau_{\frac{t}{2}}$ be the parallel transport along $\ga $ from
$p_{0}$ to $p_{t}$ and $p_{0}$ to $p_{\frac{t}{2}}$ respectively,
then for any $V \in T_{p_{0}} M$, we have $V$ and
$\tau_{\frac{t}{2}} V$ is parallel along $\overline{p_{0}
p_{\frac{t}{2}}}$, then since $s_{\frac{t}{2}}$ is an isometry,
we have $d s_{\frac{t}{2}} V$ and $d s_{\frac{t}{2}}
\tau_{\frac{t}{2}} V$ is parallel along
$\overline{p_{\frac{t}{2}} p_{t}} .$ Since $d s_{\frac{t}{2}}
\tau_{\frac{t}{2}} V=-\tau_{\frac{t}{2}} V$, we have
$d s_{\frac{t}{2}} V=d T_{t} V$ and $\tau_{\frac{t}{2}} V$ are
parallel along $\overline{p_{\frac{t}{2}} p_{t}}$, hence $d T_{t}
V=\tau_{t} V .$

Moreover, $T_{t}\left( p_{s} \right)=s_{\frac{t}{2}} s_{0}\left(
p_{s} \right)=s_{\frac{t}{2}} p_{-s}=p_{t+s}$, then $T_{t} \circ
T_{s}\left( p_{0} \right)=T_{t+s}\left( p_{0} \right)=$
$p_{t+s}$ and $d\left( T_{t} \circ T_{s} \right)_{p_{0}}=d\left(
T_{t+s} \right)_{p_{0}}$ since they are both the parallel
transport along $\gamma$, hence $T_{t+s}=T_{t} \circ T_{s}$.
\eproof
\begin{theorem}
	Let $M$ be a Riemannian globally symmetric manifold. Then the
	isometry group $I(M)$ has a smooth structure compatible with
	the compact open topology in which it is a Lie transformation
	group on $M$.
\end{theorem}
\bproof
See \cite{Hel} page 205.
\eproof



Sometimes the group $I_0(M)$, the identity component of $I(M)$,
is more useful than $I(M)$ itself, so we develop the following:

\begin{lemma}
	Suppose $G$ is a transitive Lie transformation group on a
	smooth manifold $M$, then the identity component $G_0$ is
	transitive on $M$ and $M$ is diffeomorphic to $G_0/K$, where
	$K$ is the subgroup which leaves $p$ fixed.
\end{lemma}
\bproof
$G_0$ is also transitive since the map $g \to g \cdot p$ is a
submersion and hence an open map, so the orbits of $G_0$ and
other components are open. But $M$ is connected and is the union
of these disjoint open orbits, so the orbit of $G_0$ is $M$. 
\eproof

Hence we get
\begin{corollary}
	Let $M$ be a Riemannian globally symmetric manifold and $p_0$
	be any point in $M$. If $G = I_0(M)$, and $K$ is the subgroup
	of $G$ which leaves $p_0$ fixed, then $K$ is a compact
	subgroup of the connected group $G$ and $G/K$ is
	diffeomorphic to $M$ under the mapping $gK \mapsto g \cdot
	p_0, g \in G$.
\end{corollary}

Now we have reach the step to write a Riemannian globally
symmetric space as the form $I(M)/I(M)_p \cong I_0(M)/I_0(M)_p$.
Note that the two ways of representing $M$ do not change the
tangent spaces of Lie groups, so Cartan decomposition below is
invariant whether we use $I(M)/I(M)_p$ or $ I_0(M)/I_0(M)_p$.

\begin{theorem}[Cartan decomposition]
	 Let $M$ be a Riemannian globally symmetric manifold with
	 isometry group $I(M)$ and its identity component $G =
	 I_0(M)$. Let $p_0$ be a fixed point on $M$ and $K$ is the
	 subgroup of $G$ which fixes $p_0$. Then the mapping
	 $\sigma\:g\to s_{p_0}gs_{p_0}$ is an involutive automorphism
	 of $G$, also called the Cartan involution of this symmetric
	 space.
	 
	 Let $\ssg$ and $\ssk$ denote the Lie algebras of $G$ and $K$
	 respectively. Then $\ssk = \{ X \in \ssg\:(d\sigma)_eX = X\}$
	 and  we have $\ssg = \ssk\oplus\ssp$ for $\ssp\coloneq \{X\in
	 \ssg\:(d\sigma)_eX = -X\}$. Let $\pi $ denote the natural
	 mapping $g\mapsto g\cdot p_0$ of $G$ onto $M$. Then
	 $(d\pi)_e$ maps $\ssk$ into $\{0\}$ and $\ssp$
	 isomorphically onto $T_{p_0}M$. In particular, we have
	 $T_{p_0}M\cong \ssg/\ssk\cong \ssp$ as vector spaces.
\end{theorem}
\bproof
Obviously $\sigma^2 = id$ so $\sigma$ is an involutive
automorphism of $I(M)$ and since it is well known that $H_0$ is a
normal subgroup of $H$ for any Lie group $K$, $\sigma$ maps $G$
onto itself.  If $k\in K$, the mappings $k$ and $s_{p_0}ks_{p_0}$
are isometries which induce the same mapping on $T_{p_0}M$. Then
we have $s_{p_0}ks_{p_0} = k$ for all $k\in K$ by Corollary
$\ref{11.2}$. It follows that the automorphism $(d\sigma)_e$ of
$\ssg$ is identity on $\ssk$. On the other hand, if $X\in \ssg$
is left fixed by $(d\sigma)_e$, then $s_{p_0}\exp tX s_{p_0} =
\exp tX$ for all $t$, so $X\in \ssk$. Hence $\ssk = \{X\in
\ssg\:(d\sigma)_eX = X\}$.
The direct decomposition $\ssg = \ssk\oplus\ssp$ follows from the
identity $X = \frac{1}{2}(X+d\sigma
(X))+\frac{1}{2}(X-d\sigma(X))$.

Now we prove that $\ker d\pi_e = \ssk$. Actually, if $X\in \ker
d\pi_e$,  then for any $f\in C^{\infty}(M)$, we have 
\[0 = (d\pi_eX)f = X(f\circ
\pi)=\left.\frac{d}{dt}\right|_{t=0}f(\exp tX\cdot p_0).
\]
For any $s \in \bbR$, we use the above equation on the function
$g(q)=f(\exp sX \cdot q), q\in M$. Then
\[
0=\left.\frac{d}{dt}\right|_{t=0}g(\exp tX\cdot
p_0)=\left.\frac{d}{dt}\right|_{t=s}f(\exp tX\cdot p_0
\]
which shows that $f(\exp sX\cdot p_0)$ is constant in $s$. Since
$f$ is arbitrary, we have $\exp sX\cdot p_0 = p_0$ for all $s$,
so $X\in \ssk$. On the other hand, it is clear that $d\pi_e$
vanishes on $\ssk$, hence $\ker(d\pi)_e=\ssk$ and
$\dim\mathrm{Im}(d\pi)_e = \dim\ssg-\dim\ssl = \dim G/K = \dim
M$, so $T_{p_0}M\cong \ssg/\ssk$.
\eproof


To move on, we need to learn about the $\ad$ representation and
the Killing form of Lie algebras.

\begin{definition}
	Let $G$ be a Lie group, Lie $G = \ssg$. For $\sigma \in G$,
	define an automorphism $I({\sigma})$ of Lie group $G$ as
	$I(\sigma)(g) = \sigma g\sigma^{-1}$ and an automorphism of
	Lie algebra $\ssg$ as $\ad(\sigma) = dI(\sigma)_e$.
\end{definition}
\begin{theorem}
	For any $X\in \ssg$ and $\sigma \in G$, we
	have$$\exp(\ad(\sigma)X) = \sigma\exp X\sigma^{-1}.$$
	Moreover, the map $\ad\:\sigma\mapsto\ad(\sigma)$ is a Lie
	group homomorphism from $G$ to $Gl(\ssg)$, hence $\ad$ is a
	representation of Lie group $G$.
\end{theorem}
\bproof
From lemma $\ref{31}$, $\exp((dI(\sigma))_eX) = I(\sigma)\exp X$,
so $\exp(\ad(\sigma)X) = \sigma\exp X\sigma^{-1}$. Besides, for
$\sigma_1, \sigma_2\in G$, we have
$$\exp(\ad(\sigma_1\sigma_2)X) = \sigma_1\sigma_2\exp
X\sigma_2^{-1}\sigma_1^{-1} =
\sigma_1\exp(\ad(\sigma_2)X)\sigma_1^{-1} =
\exp(\ad(\sigma_1)\ad(\sigma_2) X).$$
Hence $\ad(\sigma_1\sigma_2)X = \ad(\sigma_1)\ad(\sigma_2) X$ for
$X$ in a neighborhood of $0\in\ssg$. Since
$\ad(\sigma_1\sigma_2)$ and $\ad(\sigma_1)\circ\ad(\sigma_2)$ are
linear, we have $\ad(\sigma_1\sigma_2) =
\ad(\sigma_1)\circ\ad(\sigma_2)$. 

Finally, we prove that $\ad\: G\to Gl(\ssg)$ is smooth. It
suffices to show that for any $X\in \ssg$ and linear function $w$
on $\ssg$, the function $\sigma\mapsto w(\ad(\sigma)X)$ is smooth
at $\sigma=e$. Take a smooth function $f\in C^{\infty}(U)$ on a
neighborhood of $e$ such that $Y(f) = w(Y)$ for all $Y\in \ssg$.
Then by $\exp(t\ad(\sigma)X) = \sigma\exp tX\sigma^{-1}$, we have
$$w(\ad(\sigma)X) = (\ad(\sigma)X)f =
\left.\frac{d}{dt}\right|_{t =
0}f(\exp(t\ad(\sigma)X))=\left.\frac{d}{dt}\right|_{t=0}f(\sigma
\exp(tX)\sigma^{-1})$$
which proves the smoothness.
\eproof

\begin{definition}
	Let $\ssg$ be a Lie algebra. For any $X \in \ssg$, define a
	linear map $\mathrm{ad}X\:\ssg \to \ssg$ as
	$\mathrm{ad}X(Y)=[X,Y]$.
\end{definition}
\begin{theorem}
	For any $X\in\ssg$, $\mathrm{ad}X$ is a derivation,
	hence$$\mathrm{ad}X[Y,Z] =
	[\mathrm{ad}X(Y),Z]+[Y,\mathrm{ad}X(Z)].$$
\end{theorem}
\begin{lemma}\label{323}
Let $G$ be a Lie group,  Lie $G=\ssg$, then for any $X, Y \in
\mathfrak{g}$, we have
\[
	\exp t X \exp t Y = \exp \left\{ t(X+Y)+\frac{t^{2}}{2}[X,
	Y]+O\left(t^{3}\right) \right\}
\]
\[
	\exp t X \exp t Y \exp (-t X) = \exp \left\{ t Y+t^{2}[X,
	Y]+O\left( t^{3} \right) \right\}.
\]	
\end{lemma}
\bproof
see \cite{Hel} page 106.
\eproof

\begin{theorem}
	The map  $\mathrm{ad}\:\mathfrak{g} \rightarrow
	\mathfrak{gl}(\mathfrak{g})$ is a homomorphism of Lie
	algebras. Moreover,  $\mathrm{ad}$ is the differential of
	$\mathrm{Ad}\: {G} \rightarrow {Gl}(\mathfrak{g})$ at $e .$

\end{theorem}
\bproof
We first prove $\operatorname{ad}[X, Y]=[\operatorname{ad} X, 
\mathrm{ad} Y]$.  For any $Z \in \mathfrak{g}$, we have
$$\operatorname{ad}[X, Y](Z)=[[X, Y], Z]=(\operatorname{ad} X
\operatorname{ad} Y-\operatorname{ad} Y \operatorname{ad}
X)(Z)=[X,[Y, Z]]-[Y,[X,Z]].$$
Next we prove that the differential of $\ad$ is $\mathrm{ad}$. By
Lemma $\ref{323}$, we
have  $$\exp (\operatorname{Ad}(\exp t X) t Y) = I(\exp (t
X))(\exp t Y) = \exp \left( t Y+t^{2}[X, Y]+O\left( t^{3} \right)
\right)$$
then 
$$
\operatorname{Ad}(\exp t X) Y=Y+t[X, Y]+O\left( t^{3} \right)
$$
hence $d \operatorname{Ad}_{e}(X)=\operatorname{ad} X$.
\eproof
\begin{definition}
Let $\mathfrak{g}$ be a Lie algebra. The Killing form $B\:
\mathfrak{g} \times \mathfrak{g} \rightarrow \mathfrak{g}$ is a
bilinear symmetric form defined as
\[
B_{\mathfrak{g}}(X, Y)=\operatorname{Tr}(\mathrm{ad} {X} \circ
\mathrm{ad} {Y}).
\]	
\end{definition}

\begin{theorem}
	The Killing form is invariant under $\operatorname{Ad} \tau$
	and $\mathrm{ad} X, \forall \tau \in G$ and $X \in
	\mathfrak{g}$.
	
\end{theorem}
\bproof
Since $\sigma \coloneq  \operatorname{Ad}(\tau)$ is an automorphism of
$\mathfrak{g}$, we have (by acting on $\sigma Y$ ) that
$\operatorname{ad}(\sigma X) = \sigma \circ \operatorname{ad} X
\circ \sigma^{-1}$. Then
\[
B_{\mathfrak{g}}(\sigma X, \sigma Y) =
\operatorname{Tr}\left(\sigma \circ \operatorname{ad} X \circ
\sigma^{-1} \sigma \circ \mathrm{ad} Y \circ \sigma^{-1}\right) =
B_{\mathfrak{g}}(X, Y).
\]
Then
\[
B_{\mathfrak{g}}(\operatorname{Ad}(\exp t W) X,
\operatorname{Ad}(\exp (t W)) Y) = B_{\mathfrak{g}}(X, Y)
\]
 by taking derivative to $t$ at $t = 0$, we have
\[
B_{\mathfrak{g}}(\operatorname{ad} W(X), Y) + B_{\mathfrak{g}}(X,
\text { ad } W(Y)) = 0,\ \forall W \in \ssg.
\]
We may write it as
\[
B_{\mathfrak{g}}([W, X], Y) + B_{\mathfrak{g}}(X, [W, Y]) = 0
\]
which is easy to remember.
\eproof
From the proof we see that a bilinear form is invariant under
$\ad G$ implies its invariance under $\mathrm{ad}\ssg$.

Sometimes, the metric of a manifold may not be given. We need to
construct one in order to make it a symmetric manifold (usually
for a Lie group).

\begin{lemma}
Let $G$ be a compact Lie group, Lie $G = \ssg$, then
$\mathfrak{g}$ admits an inner product $\langle\cdot,
\cdot\rangle$ such that it is invariant under $\mathrm{Ad} {G}$.
As a consequence, the
Killing form of $G$ is semi-definite negative and $B_{\ssg}(X, X)
= 0$ if and only if $X$ is in the center of $\mathfrak{g}$.	
\end{lemma}
\bproof
Let $g_{0}$ be any inner product on $\mathfrak{g}$ and it induces
a right-invariant metric on
$G$, then we define
\[
g(V, W) = \int_{G} g_{0}(\operatorname{Ad}(x) V,
\operatorname{Ad}(x) W) d x
\]
so $g$ is a inner product on $\mathfrak{g}$ and
\[
\begin{aligned}
	g(\operatorname{Ad}(y) V, \operatorname{Ad}(y) W) & =
	\int_{G} g_{0}(\operatorname{Ad}(x y) V, \operatorname{Ad}(x
	y) W)\mathrm{vol}_{g_{0}} \\
	& = \int_G R_{y}^{*}\left( g_{0}(\operatorname{Ad}(x) V,
	\operatorname{Ad}(x) W) \text { vol}_{g_{0}} \right) = g(V,
	{W})
\end{aligned}
\]
hence $g$ is $\operatorname{Ad}(G)$ invariant. Then
\[
g(\operatorname{Ad}(\exp t W) X, \operatorname{Ad}(\exp (t W)) Y)
= g(X, Y)
\]
then by taking derivative to $t$ at $t=0$, we have
\[
g(\operatorname{ad} W(X), Y)+g(X, \text { ad } W(Y)) = 0
\]
hence ad$W$ is represented as a skew-symmetric matrix $A$ with
respect to an
orthonormal basis of $\mathfrak{g}$, then we have
\[
B_{\ssg}(X, X) = \operatorname{Tr}(\operatorname{ad} X \circ
\operatorname{ad} X)=\operatorname{Tr}\left(A^{2}\right) =
\operatorname{Tr}\left(-A A^{T}\right) \leq 0
\]
and the equality holds if and only if ad $X=0$, hence $X$ is in
the center of $\mathfrak{g}$.
\eproof

Then $-B$ is an $\operatorname{Ad}(G)$ invariant inner product on
$\mathfrak{g}$ and it can be extended to a bi-invariant
Riemannian metric on $G$ as follows. We define a metric on $G$ by
\[
\langle X, Y\rangle_{g} = \left\langle
d\left(L_{g^{-1}}\right)_{g} X, d\left( L_{g^{-1}} \right)_{g}
Y\right\rangle
\]
Then $L_{g}$ is naturally an isometry. $R_{g}$ is also an
isometry by 
\[
\begin{aligned}
	\left\langle d\left (R_{g} \right)_{e} X, d\left( R_{g}
	\right)_{e} Y\right\rangle_{g} & = \left\langle d\left(
	L_{g^{-1}} \right)_{g} d\left( R_{g} \right)_{e} X, d\left(
	L_{g^{-1}} \right)_{g} d\left( R_{g} \right)_{e}
	Y\right\rangle \\
	&=\left\langle\operatorname{Ad}\left(g^{-1}\right) X,
	\operatorname{Ad}\left( g^{-1} \right) Y\right\rangle=\langle
	X, Y\rangle.
\end{aligned}
\]


	In the following, we deal with simply-connected Riemannian
	symmetric manifolds. Now let $M$ be a simply-connected
	Riemannian symmetric manifold represented as $G / K$, where
	$G=I_{0}(M)$ and $K$ is the compact subgroup $I_0(M)_p$ and
	we have the canonical decomposition of $\mathfrak{g} =
	\mathfrak{k}\oplus\mathfrak{p}$,
	where $\ssk$ is the Lie algebra of $K$ and $\mathfrak{p}$ 
	the $-1$ eigenspace of the differential
	$d \sigma_{e}$ of the involutive automorphism $\sigma
	\: g \mapsto s_{p_{0}} g s_{p_{0}} .$ We have

\begin{lemma}\label{229}
		The vector space $\mathfrak{p}$ is invariant under
		$\operatorname{Ad} K$. In other words, $\mathfrak{p}$ is
		a representation of the Lie group $K$.
\end{lemma}
\bproof
For any $X \in \mathfrak{p}$ and $k \in K$, we have $\exp
(\operatorname{Ad}(k) t X) = k \exp t X k^{-1}$, then
\[
\sigma(\exp \operatorname{Ad}(k) t X) = s_{0} k \exp t X k^{-1}
s_{0} = k s_{0} \exp t X s_{0} k^{-1}.
\]
Since $X \in \mathfrak{p}$, we have $\left.\frac{d}{d t} s_{0}
\exp t X s_{0}\right|_{t=0} = d\sigma_{e}X = -X$, then
\[
\begin{aligned}
	d \sigma_{e}(\operatorname{Ad}(k) X) & = \left.\frac{d}{d t}
	\sigma(\exp \operatorname{Ad}(k) t X)\right|_{t=0} \\
	& = \left.\frac{d}{d t}\right|_{t=0} I(k)\left(s_{0} \exp t
	Xs_{0}\right) = \operatorname{Ad}(k)(-X),
\end{aligned}
\]
hence $\operatorname{Ad}(k) X \in \mathfrak{p}$.
\eproof

\begin{proposition}
	Let $\mathfrak{g} = \mathfrak{k}\oplus\mathfrak{p}$ be
	defined as above, then we have
	\[
	[\mathfrak{k}, \ssk] \subset \mathfrak{k},
	\quad[\mathfrak{p}, \mathfrak{p}] \subset \mathfrak{k},
	\quad[\mathfrak{k}, \mathfrak{p}] \subset \mathfrak{p}.
	\]
	As a consequence, we have $\mathfrak{k}$ and $\mathfrak{p}$
	are orthogonal with respect to the Killing
	form of $\mathfrak{g}$. 
\end{proposition}
\bproof
$[\mathfrak{k}, \ssk] \subset \mathfrak{k}$ is obvious since
$\ssk$ is a Lie subalgebra of $\ssg$. The other two is clear as
$d\sigma_{e}$ is a Lie algebra homomorphism. $[\mathfrak{k},
\mathfrak{p}] \subset \mathfrak{p}$ can also be implied from
lemma $\ref{229}$.  Then for any $X\in \ssk$ and $Y\in \ssp$,
$\mathrm{ad}X \circ \mathrm{ad}Y$ maps $\ssk$ to $\ssp$ and
$\ssp$ to $\ssk$. Hence $B(X, Y) = 0$.
\eproof

Next we turn to look at the more geometric ideas.

If $G$ acts by isometries on a manifold $M$, we can associate to
each $X \in \mathfrak{g}$ a vector field $X^{*}$ on $M$ called an
action field which is defined by
\[
X^{*}(p) = \left.\frac{d}{d t}\right|_{t=0}(\exp (t X) \cdot p) .
\]
Notice that the flow of $X^*$ acts by isometries, $X^*$ is a
Killing vector field. We should be careful that $\left[X^{*},
Y^{*}\right] = -([X, Y])^{*}$ since  the flow of $X \in
\mathfrak{g}$ is given by right translation, different from that
of $X^*$. 

We can then identify
\[
\mathfrak{p} \simeq T_{p_{0}} M \text { via } X \rightarrow
X^{*}\left(p_{0}\right)
\]
This is an isomorphism since $X^{*}\left(p_{0}\right) = 0$ iff $X
\in \mathfrak{k}$.

\begin{proposition}\label{geo}
	Let $G = I_0(M), K = I_0(M)_p$ such that $M\cong G/K$, with
	Cartan decomposition $\mathfrak{g} = \mathfrak{k} \oplus
	\mathfrak{p}$.
	
	(a) For any vector field $Y$ on $G / K$ and $X \in
	\mathfrak{p}$, we have $\left( \nabla_{X^*}  Y \right)\left(
	p_{0} \right)=$
	$\left[ X^{*}, Y \right]\left( p_{0} \right) .$
	
	(b) If $X, Y, Z \in \mathfrak{p}$, then $\left( R\left(
	X^{*}, Y^{*} \right) Z^{*} \right)\left( p_{0} \right)=-[[X,
	Y], Z]^{*}\left( p_{0} \right)$.
\end{proposition}
\bproof
(a) For $X \in \mathfrak{p}$, let $\gamma(t)=(\exp t X) \cdot
p_{0}$ , which is a geodesic in $M$. We have the corresponding
transvection $T_{t}=s_{\gamma(\frac{1}{2})} s_{\gamma(0)} =
{L_{\exp t X}}$ which is the flow of $X^*$.  Also $\left(d
T_{t}\right)_{\gamma(s)}$
 is parallel translation along $\gamma$.  Thus if $ Y$  is any
 vector field on $M$, we have  $\nabla_{X^*} Y = \frac{d}{d
 t}|_{t=0} \left( P_{t}^{-1} Y(\gamma(t)) \right) = \frac{d}{d
 t}|_{t=0}\left( d T_{t}^{-1} \right)_{\gamma(t)}
 Y(\gamma(t))=\left[ X^{*}, Y \right]$.
 
 (b) We first compute (cverything at $p_{0}$ )
 \[
 \nabla_{X^*} \nabla_{Y^*} Z^{*} = \left[X^{*}, \nabla_{Y^*}
 Z^{*}\right] = \nabla_{\left[X^{*}, Y^*\right]}
 Z^{*}+\nabla_{Y^*}\left[X^{*}, Z^{*}\right]
 \]
 since isometries preserve the connection and the flow of $X^{*}$
 consists of isometries. Since $[\ssp, \ssp] \subset
 \mathfrak{k}$ we have $[X, Y]^{*}\left(p_{0}\right)=0$ and hence
 \[\nabla_{X^*} \nabla_{Y^*} Z^{*}=\nabla_{Y^{*}} \left[X^{*},
 Z^{*}\right]=-\nabla_{Y^*}[X, Z]^{*}=-\left[Y^{*},[X,
 Z]^{*}\right]=[Y,[X, Z]]^{*}
 \]
 Thus
 \[
 \begin{aligned}
 	R\left(X^{*}, Y^{*}\right) Z^{*} &=\nabla_{X^*} \nabla_{Y^*}
 	Z^{*}-\nabla_{Y^*} \nabla_{X^*} Z^{*}-\nabla_{\left[X^{*},
 	Y^{*}\right]} Z^{*} \\
 	&=[Y,[X, Z]]^{*}-[X,[Y, Z]]^{*}=-[[X, Y], Z]^{*}
 \end{aligned}
 \]
 by the Jacobi identity.
\eproof

  We usually simply state
 \[
 \nabla_{X} Y = [X, Y], \quad R(X, Y) Z = -[[X, Y], Z]
 \]
 with the understanding that this only holds at $p_{0}$.

\begin{remark}
	Part (a) gives rise to a geometric interpretation of 
	Cartan decomposition in terms of Killing vector fields,
	assuming that $G={I}_{0}(M)$ :
	\[
	\mathfrak{k} = \left\{X \in \mathfrak{g} \mid X^{*}\left(
	p_{0} \right) = 0 \right\} \quad \mathfrak{p} = \left\{X \in
	\mathfrak{g} \mid \nabla_{v} X^{*}\left(p_{0}\right) = 0
	\text { for all } v \in T_{p_{0}} M \right\}
	\]
	The first equality is obvious and for the second one, we
	observe that (a) implies that $\left( \nabla_{X^{*}} Y^{*}
	\right)\left( p_{0} \right) = [X, Y]^{*}\left( p_{0}
	\right)=0$ for $X, Y \in \mathfrak{p} \operatorname{\ since\
	}[\mathfrak{p}, \mathfrak{p}] \subset \mathfrak{k}$.
	Then the equality holds for the reason of dimension. 
\end{remark}

\begin{corollary}\label{geo2}
	For any $X, Y, Z \in T_{p} M$, we have
	\[
		R(X, Y) Z = [Z,[X, Y]] 
	\]
	\[
		\operatorname{Ric}(Y, Z) = -\frac{1}{2} B(Y, Z)
	\]
\end{corollary}
\bproof
See \cite{Peter} Theorem 10.1.6.
\eproof
\subsection{de Rham decomposition}

Next we consider the classification of  Riemannian (globally)
symmetric spaces.

For a Riemannian manifold and a fixed point $p$ one defines the
holonomy group $\mathrm{Hol}_{p} = \left\{P_{\gamma} \mid
\gamma(0) = \gamma(1) = p\right\}$ given by parallel translation
along piecewise smooth curves, and we let $\mathrm{Hol}_{p}^{0}$
be its identity component. Note that for different point $p$, the
group are isomorphic with each other. Thus We  often denote it by
Hol. It has some fundamental properties :
\begin{theorem}
	Assume that $M$ is complete. Then
	
	(a) $\mathrm{Hol}$ is a Lie group with  $\mathrm{Hol}^{0}$
	compact.
	
	(b) $\mathrm{Hol}^0$ is given by parallel translation along
	null homotopic curves.
	
	(c) $\mathrm{Hol}_{p}$ is connected if $M$ is simply
	connected,
\end{theorem}
\bproof
See \cite{Ziller} Theorem 6.7.
\eproof

If we write $M = G / K$, $G = I(M)$, $K = I(M)_p$,  notice that
$\mathrm{Hol}_{p}$ is a subgroup of $\mathrm{O}\left(T_{p}
M\right)$ by definition, as is $K$ via the isotropy
representation.

\begin{corollary}
	If $M = G / K$ is a symmetric space with $G = {I}(M)$, then
	$\operatorname{Hol}_{p} \subset K$.
\end{corollary}
\bproof
See \cite{Ziller} Corollary 6.8.
\eproof


We can now combine this  with the de Rham decomposition theorem,
which is an significant application of holonomy groups, 
Recall that:
\begin{definition}
	$M$ is said to be decomposable if $M$ can be written as
	$(M,g) = (M_1,g_1)\times(M_2,g_2)$. Otherwise $M$ is called
	indecomposable.
\end{definition}
\begin{theorem}[de Rham]
	 Let $M$ be a simply connected Riemannian manifold,
	 $\mathrm{Hol}_{p}$ is the holonomy group for some $p\in M$.
	 Let $T_{p} M = W_{0} \oplus W_{1} \oplus \cdots \oplus
	 W_{k}$ be a decomposition such that each $W_i$ is
	 irreducible with respect to $\mathrm{Hol}_{p}$ , and that 
	 $W_{0} = \left\{ v \in T_{p} M \mid h v = \right.$ $v$ for
	 $\forall \left.h \in \mathrm{Hol}_{p} \right\}$. 
	 
	 Then $M = N_{0} \times \cdots \times N_{k}$, where $N_{0}$
	 is isometric to flat $\mathbb{R}^{n}$. If $p = \left(p_{0},
	 p_{1}, \ldots, p_{k} \right)$, then $T_{p_{i}} N_{i} \simeq
	 W_{i}$ and $N_{i}$ is indecomposable for $i \geq 1$.
	 Moreover, the decomposition is unique up to order and
	 $\mathrm{Hol}_{p} \simeq \mathrm{Hol}_{p_{1}} \times \cdots
	 \times \mathrm{Hol}_{p_{k}}$ with $\mathrm{Hol}_{p_{i}}$ the holonomy
	of $N_{i}$ at $p_{i} .$ At last we have, ${I}_{0}(M) = {I}_{0}\left(N_{0}\right) \times \cdots \times {I}_{0}\left(N_{k}\right)$.
\end{theorem}
\bproof
See \cite{Ziller} Theorem 6.9.
\eproof

Since for a symmetric space Hol$_{p} \subset K$, this implies 

\begin{corollary}
	If the Riemannian symmetric manifold $M = G / K$ is simply
	connected with $M$ indecomposable, then $K$ acts irreducibly
	on the tangent space.
\end{corollary}

This motivates the definition of reducibility:

\begin{definition}
	A symmetric manifold $G / K$ with $G = {I}(M)$, $K = I(M)_p$,
	is called irreducible if $K_0$ (the identity component of
	$K$) acts irreducibly (by $\ad$) on $T_pM$, and reducible
	otherwise.
\end{definition}

Notice that the definition does not change if we replace $G$ by
$G = I_0(M)$ since $I(M)/K\cong I_0(M)/I_0(M)_p$ and $I_0(M)_p =
I_0(M)\cap I(M)_p$ has the same Lie algebra with $I(M)_p$, just
as we said in the Carton decomposition part.

Also, the definition is independent of different  $p$, since the
isometry
group acts transitively on $M$ and for any point $q$ there exists
an isometry $\varphi \in I(M)$ such that $\varphi(p) = q$, then
$\varphi G_{p}^{0} \varphi^{-1} = G_{q}^{0}$ and the reduciblity
of the action $G_{p}^{0}$ on $T_{p} M$ is the same as that of
$G_{q}^{0}$. Equivalently, one can use $G = I(M)$ and $K = G_{p}
$ since the definitions are
both equivalent to the reduciblity of the action of
$\mathfrak{k}$ on $\mathfrak{p}$ by ad.


By the above, we have
\begin{theorem}
	If $M$ is a simply connected symmetric manifold and $M$ is
	indecomposable, then $M$ is irreducible.
\end{theorem}
\bproof
Since $M = I_{0}(M) / K$ and $M$ is simply-connected, then $K$ is
connected by the homotopy sequence
\[
\begin{aligned}
	\pi_{1}(K) & \rightarrow \pi_{1}(G) \rightarrow \pi_{1}(G /
	K) \rightarrow e \pi_{0}(K) \\
	& \rightarrow \pi_{0}(G) \rightarrow \pi_{0}(G / K) \rightarrow 1
\end{aligned}
\]
where $\pi_{0}$ denotes the set of connected components, but in the case $\tilde{G}$ and
$\tilde{K}$ are Lie groups, these sets are all groups with natural multiplication. So $K = K_{0} .$ Then $\mathrm{Hol}_{p} \subset K_{0}$, so $\mathrm{Hol}_{p}$ acts irreducibly on the tangent space indicates that $K_{0}$ acts irreducibly.
\eproof

However the other side is wrong even if it is simply connected.
Just look at flat $\mathbb{R}^{n}$, we have $K = \mathrm{O}(n)$
which acts irreducibly. But this is  the only exception by the
conclusion to be learned: If $M = G / K$ is irreducible, then $M
= \mathbb{R}^{n} \times N$ with the product metric, where the
metric on $\bbR^n$ is flat and $N$ has a metric making it a
symmetric manifold and is decomposable.

 The DeRham decomposition theorem implies:
 
 \begin{corollary}
 	If the symmetric manifold $M = G / K$ is  simply connected,
 	then $M$ is isomorphic to product of irreducible symmetric
 	manifolds $M_1\times \cdots\times M_k$.
 \end{corollary}
\bproof
See \cite{Ziller} Corollary 6.12.
\eproof
 
 Another important consequence:
 \begin{corollary}
 	If the symmetric manifold $M = G / K$ is simply connected
 	and $G$ is simple, then $M$ is irreducible.
 \end{corollary}
\bproof
It suffices to show $M$ is indecomposable. By contradiction, if
$M = M_{1} \times \cdots \times M_{k}$, then $I_{0}(M) =
{I}_{0}\left( M_{1} \right) \times \cdots \times {I}_{0}\left(
M_{k} \right)$
so that $G$ is not simple. 
\eproof
In fact, the converse is also true  except for one example.

One easily sees:
\begin{corollary}\label{333}
	An irreducible symmetric manifold is Einstein, i.e.\
	$\Ric(g) =  \lambda g$ for some constant $\lambda$.
	Moreover, the metric is unique up to scaling.
\end{corollary}
\bproof
This follows from the following useful lemma.
\eproof
\begin{lemma}
	Let $B_{1}, B_{2}$ be two symmetric bilinear forms on a
	vector space $V$ and let $B_{1}$ be positive definite. If a
	compact Lie group $K$ acts irreducibly on $V$ such that
	$B_{1}$ and $B_{2}$ are invariant under $K$, then $B_{2} =
	\lambda B_{1}$ for some constant $\lambda$.
\end{lemma}
\bproof
By the non-degeneration of $B_{1}$, there is an endomorphism
$\phi\: V \rightarrow V$ such that $B_{2}(u, v) = B_{1}(\phi u,
v) .$ As $K$ acts by isometries, $B_{1}(k \phi u, v) =
B_{1}\left( \phi u, k^{-1} v \right) = B_{2}\left( u, k^{-1} v
\right)=B_{2}(k u, v) = B_{1}(\phi k u, k v)$ and therefore
$\phi k = k \phi$
for all $k \in K$. Besides, by symmetry we have $B_{1}(\phi u,
v) = B_{2}(u, v) = B_{2}(v, u) = B_{1}(\phi v, u) = B_{1}(u,
\phi v)$. So $\phi$ is symmetric with respect to $B_{1}$ and the
eigenvalues of $\phi$ are real. Suppose $W \subset V$ is an
eigenspace of $\lambda$, then $W$ is invariant under $K$ since
$k \phi=\phi k$. As $K$ acts irreducibly, $W = \{0\}$ or $W =
V$. Thus there is a constant $\lambda$ such that $\phi =
\lambda$ Id and hence $B_{2} = \lambda B_{1}$. Notice that
$\lambda \neq 0$ since otherwise
$B_{2} = 0$.
\eproof

\bproof
Return to Corollary $\ref{333}$. From above, we see that the
metric is unique up to scaling. Since isometries preserve the
curvature, Ric is also a symmetric bilinear form invariant under
$K$, hence at any point there exists smooth function $\lambda$
such that $\Ric = \lambda g$. Since $\Ric $ is parallel,
$\lambda$ has to be a constant.
\eproof
\subsection{Classification theorems}
\begin{definition}
	Let $(M,g)$ be a Riemannian symmetric manifold. If the Ricci
	curvature is positive(negative), then $M$ is called of
	(non-)compact type. If the Ricci curvature is zero, then $M$
	is called of Euclidean type.
\end{definition}

\bremark
Note that this only gives a classification irreducible ones.
Also, since $\Ric$ is parallel, we have that the eigenvalues of
$\Ric$ are the same. So by Myer's theorem, $M$ is compact if it
is of compact type, and by Bochner formula, $M$ is non-compact
if $M$ is of non-compact type (see \cite{Peter} Corollary 8.2.5
for example).

\eremark

When $M$ is of Euclidean type, we refer to this theorem:
\begin{theorem}
	If a Riemannian symmetric manifold is Ricci flat, then it is
	flat.
\end{theorem}
\bproof
See \cite{Peter} Corollary 8.2.5.
\eproof
 Then if $M$ is of Euclidean type and  is simply connected, then
 it must be isomorphic to $\bbR^n$ up to uniformization. If not
 simply connected, we will show that $M$ is covered by $\bbR^n$.
 As a matter of fact, we also have that any symmetric manifold
 of (non-)compact type can be covered by a product of
 (non-)compact irreducible symmetric manifolds, which will be
 shown in the following:
 
 \begin{theorem}\label{cover}
 	 Let $(M, g)$ be a Riemannian symmetric manifold represented
 	 as $G / K$, where $G = I_{0}(M), K = I_0(M)_p$  and
 	 $\mathfrak{g} = \mathfrak{k}\oplus\mathfrak{p}$ is the
 	 Cartan decomposition with $\sigma\: \mathfrak{g}
 	 \rightarrow \mathfrak{g}$ as the Cartan involution. Let
 	 $\tilde{G}$ be the simply connected Lie group induced by
 	 $\mathfrak{g}$. Then there exists a connected Lie subgroup
 	 $\tilde{K}$ corresponding to $\mathfrak{k}$ and a
 	 $\tilde{G}$ -invariant metric on $\tilde{M}\coloneq \tilde{G} /
 	 \tilde{K}$ such that $\tilde{M}$ is a Riemannian symmetric
 	 manifold and moreover, it is a Riemannian covering of $M$.
 \end{theorem}
To prove this, we first give the lemma:
\begin{lemma}\label{11}
	Let $G$ be a Lie group and $H$ be a connected Lie subgroup
	with Lie algebras $\mathfrak{g}, \mathfrak{k}$ and
	$\mathfrak{p}$ is a subspace of $\mathfrak{g}$ such that
	$\mathfrak{g} = \mathfrak{k}\oplus\mathfrak{p}$ and
	$\mathrm{Ad} H(\mathfrak{p}) \subset \mathfrak{p}$. Then
	there is a one-to-one correspondence between the set of all
	$G$-invariant
	metrics on the quotient manifold $G / H$ and the set of all
	inner products $Q$
	on $\mathfrak{p} \simeq T_{o}(G / H)$ such that
	\[
	Q(\operatorname{Ad}(h)(X), \operatorname{Ad}(h)(Y)) = Q(X,
	Y), \forall h \in H ,\ X, Y \in \ssp.
	\]
\end{lemma}
\bproof
On one hand, suppose $G / H$ has a G-invariant metric $g$. Then
let $Q$ be the inner product on $\mathfrak{p} \simeq T_{o}(G /
H)$ induced by $g$. Since the adjoint representation of $H$ on
$\mathfrak{p}$ and the isotropy representation of $H$ on
$T_{o}(G / H)$ is the same under $\mathfrak{m} \simeq T_{o}(G /
H)$, then since $g$ is $G$ -invariant, we have
\[
Q(d h(X), d h(Y)) = Q(X, Y) . \forall h \in H, \forall X, Y \in
T_{o}(G / H),
\]
and hence
\[
Q(\operatorname{Ad}(h)(X), \operatorname{Ad}(h)(Y)) = Q(X, Y).
\forall h \in H, \forall X, Y \in \mathfrak{p}.
\]
On the other hand, given an inner product $Q$ on $\mathfrak{p}$
invariant under $\operatorname{Ad} H$, it
is an inner product on $T_{o}(G / H)$ which is invariant under
the pushforward by $H$, then define
$$
g_{a H}(u, v)\coloneq Q\left(\left(d \tau_{a}\right)_{a
H}^{-1}(u),\left(d \tau_{a}\right)_{a H}^{-1}(v)\right), \forall
u, v \in T_{a H}(G / H)
$$
we have a well-defined G-invariant metric on $G / H$.
\eproof
\bproof
Back to theorem $\ref{cover}$. We prove that $\tilde{M}$ is a
Riemannian symmetric manifold.  Let $\sigma\: \mathfrak{g}
\rightarrow \mathfrak{g}$ be the canonical involutive
automorphism. By the theory of simply-connected
Lie groups, there exists an involutive automorphism of Lie
groups $\tilde{\sigma}$ such that $d \tilde{\sigma}_{e} =
\sigma$. Take $\tilde{K}$ to be $\tilde{G}_{\sigma}^{0}$, which
is the identity component of the group $\tilde{G}_{\sigma}$ of
fixed point in $\tilde{G}$ by $\tilde{\sigma}$. Then
\[
\text { Lie } K = \text { Lie } \tilde{G}_{\sigma}^{0} =
\operatorname{Lie} \tilde{G}_{\sigma} = \{X \in \mathfrak{g}
\mid \sigma(X) = X\} = \mathfrak{k} \text { . }
\]
Besides, since $\tilde{K}$ is connected, we have that
\[
Q(\operatorname{Ad}(k) X, \operatorname{Ad}(k) Y)=Q(X, Y),
\forall k \in \tilde{K}, X, Y \in \mathfrak{p}
\]
if and only if
\[
Q([Z, X], Y)+Q(X,[Z, Y]) = 0, \forall Z \in \mathfrak{k}, X, Y
\in \mathfrak{p}
\]
which is due to $Q$ is the inner product of $M = G / K$ at $0 =
e K$.

So by Lemma $\ref{11}$, we have that $Q$ induces a $\tilde{G}$
-invariant metric on $\tilde{G} / \tilde{K}$. Next we prove that
$\tilde{M}$ endowed with the metric is a Riemannian symmetric
manifold. Indeed, since $K = \tilde{G}_{\sigma}^{0} \subset
\tilde{G}_{\sigma}$, we have there exists a map
$\sigma_{0}\:\tilde{G} / \tilde{K} \rightarrow \tilde{G} /
\tilde{K}$ such that the diagram commutes,
\[
\begin{tikzcd}
  G\ar[r, "\pi"] \ar[d, "\sigma"] &G/K \ar[d, "\sigma_0"]\\
  G\ar[r, "\pi"] &G/K
\end{tikzcd}
\]
hence
\[
\sigma_{0}(a \tilde{K})=\tilde{\sigma}(a) \tilde{K}, \forall a
\in \tilde{G}
\]
then $\sigma_{0}(e \tilde{K}) = \sigma_{0}(\pi(e)) =
\pi(\tilde{\sigma}(e)) = \pi(e) = \{\tilde{K}\}$, hence $0 = e
K$ is a fixed
point of $\sigma_{0}$ and it is clear that $\sigma_{0}^{2}= i
d_{\tilde{G} / \tilde{K}}$. Then for any $g \tilde{K} \in
\tilde{G} / \tilde{K}$, let $\tau_{g}\: \tilde{G} / \tilde{K}
\rightarrow \tilde{G} / \tilde{K}$ be defined as $\tau_{g}(a
\tilde{K}) = (g a) \tilde{K}, \forall g \in \tilde{G}$, and let
$\sigma_{g \tilde{K}}\coloneq  \tau_{g} \sigma_{0} \tau_{g}^{-1}$,  it
is clear that the definition is independent of the choice of $g$
and $\sigma_{g K}^{2} = i d_{\tilde{G}} / \tilde{K}$.

Now we can prove that $x = g \tilde{K}$ is an isolated fix point
of $\sigma_{x}$ and $d \sigma_{x} = -i d$ on $T_{x}(G / K)$ and
$\sigma_{x}^{*} g = g$, hence $\tilde{M}$ is a Riemannian
symmetric manifold. By homotopy theory, we have a long exact
sequence
\[
\begin{aligned}
	\pi_{1}(\tilde{K}) & \rightarrow \pi_{1}(\tilde{G})
	\rightarrow \pi_{1}(\tilde{G} / \tilde{K}) \rightarrow
	\tilde{\pi}_{0}(\tilde{K}) \\
	& \rightarrow \pi_{0}(G) \rightarrow \pi_{0}(G / K)
	\rightarrow 1.
\end{aligned}
\]
So since $G$ is connected and simply-connected, $\tilde{K}$ is
connected, we have $\tilde{G} / \tilde{K}$ is simply-connected.
Finally, we prove that $\tilde{G} / \tilde{K}$ is a Riemannian
covering of $G / K$. Let $\varphi$ be the homomorphism of
$\tilde{G}$ to $G$ such that $(d \varphi)_{e}\: \mathfrak{g}
\rightarrow \mathfrak{g}$ is identity map and let $\psi\:
\tilde{G} / \tilde{K}$ be defined as $g \tilde{K} \rightarrow
\varphi(g) K$.
Since the metric of $\tilde{G} / \tilde{K}$ is got by
translating the metric of $G / K$ at $e K$, we have that the
mapping $\psi\: g \tilde{K} \rightarrow \varphi(g) K$ is a local
isometry.

By the following result in Riemannian geometry, we have that
$\psi$ is a Riemannian covering:

 Let $M, N$ be Riemannian manifolds and $M$ is complete. Then
 any local
isometry map $\varphi\: M \rightarrow N$ is a Riemannian
covering map.
\eproof
Hence we get the theorem:
\begin{theorem}
	Let $M$ be a Riemannian symmetric manifold of (non)-compact
	type, then the Riemannian universal covering $\tilde{M}$ of
	$M$ is a product of (non)-compact irreducible symmetric
	manifolds. If $M$ is of Euclidean type, then the Riemannian
	universal covering is an Euclidean space.
\end{theorem}

So we have the equivalent definitions of types which occurs in
many articles:

\begin{definition}
	Let $(M, g)$ be a Riemannian symmetric manifold and
	$(\tilde{M}, \tilde{g})$ is a universal Riemannian covering
	manifold of $M$. Suppose $\tilde M = \tilde M_{1} \times
	\cdots \times \tilde M_{k}$ is the decomposition of products
	of irreducible symmetric spaces. If $\tilde M_{i}$'s are
	all of (non)-compact (Euclidean) type, then $M$ is called of
	(non)-compact
	(Euclidean) type.
\end{definition}
and
\begin{definition}
	Let $(M, g)$ be a Riemannian symmetric manifold represented
	as $G / K$, where $G = I_{0}(M), K = I_0(M)_p$ and
	$\mathfrak{g} = \mathfrak{k}\oplus\mathfrak{p}$ is the
	canonical decomposition. If $\left.B\right|_{\ssp}$ is
	negative definite, then $M$ is called of compact type; if
	$\left.B\right|_{\ssp}$ is positive definite, then $M$ is
	called of non-compact type; if $\left.B\right|_{\ssp} = 0$,
	then $M$ is called of Euclidean type. The manifold of
	compact type or non-compact type is called of semisimple
	type.
\end{definition}

\begin{proposition}
	Let $M$ be a symmetric of compact type or non-compact type,
	then $I(M)$ is semisimple.
\end{proposition}
\bproof
See \cite{Ziller} Proposition 6.38.
\eproof
\begin{proposition}
	If $M$ is a symmetric manifold of (non)-compact type, then
	$\sec \geq(\leq) 0$.
\end{proposition}
\bproof
It suffices to prove for irreducible ones, in which case
$B=\lambda\langle\cdot,\cdot\rangle$ for some
$\lambda \neq 0$. Then since $R(X, Y) Z=[Z,[X, Y]]$, we have
\[
\begin{aligned}
	\lambda \sec (u, v) & = \lambda\langle R(u, v) v, u\rangle
	\\
	&=-\lambda\langle[[u, v], v], u\rangle=-B([[u, v], v], u) \\
	&=B([u, v],[u, v]) < 0.
\end{aligned}
\]
The last inequality is because $B|_{\ssk}<0$, we can also refer
to \cite{Ziller} Proposition 6.38 proof of (c) and (d).
\eproof
This implies in particular that for a non-compact type $M$, $G =
I_0(M)$ is simple.  
\begin{theorem}
	Let $M$ be a Riemannian symmetric manifold of non-compact
	type, then $M$
	is diffeomorphic to $\mathbb{R}^{n}$ and in particular, $M$
	is simply-connected.
\end{theorem}
\bproof
Since sec $\leq 0$, by Hadamard theorem, we have $\exp _{p}$ is
a local diffeomorphism for any $p \in M$. By Hopf-Rinow theorem,
we have $\exp _{p}$ is surjective. So it suffices to prove that
$\exp _{p}$ is injective. Write $M$ as $M = G / K$, $G = I(M),
K=G_{p}$, then for any $X \in T_{e K} M$, it corresponds to a
Killing vector field $X^{*}$, which is an element in
$\mathfrak{g}$ and $\left\{T_{t}\right\}$ is the one-parameter
subgroup of $G$ with $\left.\frac{d}{d t}\right|_{t=0} = X^{*}$,
hence $T_{t} = \exp \left(t X^{*}\right)$, where exp is the
exponential map of the Lie group $G$.

Since $T_{t} p$ is the geodesic starting at $p$ with direction
$X$ and $M \simeq G / K$ is the same $G$-set, we have that the
geodesic is $\exp \left(t X^{*}\right) K$ and hence the
exponential map on $M$ at $p$ is $X \rightarrow \exp
\left(X^{*}\right) K$. To show the injectivity, suppose $X, W
\in T_{p} M$ satisfies that there exists $h \in K$ such that
$\exp (X) h = $ $\exp \left(W\right)$. Then
\[
\operatorname{Ad}(\exp (X)) \operatorname{Ad}(h) =
\operatorname{Ad}\left(\exp \left(W\right)\right)
\]
and $e^{\mathrm{ad} X} \operatorname{Ad}(h) =
e^{\mathrm{ad}\left(W\right)}$. Let $B^{*}(X, Y)\coloneq -B(\sigma(X),
Y)$, where $\sigma$ is the
involution. Since $M$ is of non-compact type, we have
$\left.B\right|_{\ssp} = -2 \mathrm{Ric}>0$. Since
$\left.\sigma\right|_{\mathfrak{k}} = i d$ and
$\left.\sigma\right|_{\mathfrak{p}} =-i d$ and
$\left.B\right|_{\ssk}<0$, we have $B^{*}$ is positive definite.

Then for $X \in \mathfrak{p}$, we have
\[
\begin{aligned}
	B^{*}(\operatorname{ad}(X) Z, Y)
	&=-B(\sigma(\operatorname{ad} X(Z)), Y) \\
	&=-B([\sigma X, \sigma Z], Y) \\
	&=B([X, \sigma(Z)], Y) \\
	&=-B(\sigma(Z),[X, Y])=B^{*}(Z, \operatorname{ad} X(Y)),
\end{aligned}
\]
hence ad$X$ is symmetric when $X \in \mathfrak{\ssp}$. Then
$e^{\mathrm{ad} X} \operatorname{Ad}(h)$ and $e^{\mathrm{ad} W}$
are both polar decompositions of a linear map, hence
$\mathrm{ad} X = \mathrm{ad}W$ and $X-W \in$ $Z(\mathfrak{g})=0$
by the semisimple property of $G$. Then $X=W$, hence $\exp _{p}$
is injective, hence an diffeomorphism.
\eproof

The above theorems give  a general classification and relevant
properties of symmetric manifolds.

Of course, we can classify them in a more subtle way. Usually the
case of compact type and non-compact type are studied
separately. We have the following  in order to know better:

\begin{proposition}
	A non-compact irreducible symmetric space is a quotient ${G}
	/ {K}$ where $G$ is a real simple non compact Lie group with
	trivial center and $K$ is a maximal compact subgroup of $G$.
\end{proposition}
\bproof
See \cite{Besse} Corollary 7.79.
\eproof
\begin{remark}
	 Usually these symmetric spaces are called "of type III" if
	 $G$ is absolutely simple (i.e., if the complexification
	 $g^{\bbC} = g \otimes \mathbb{C}$ is simple as a complex Lie
	 algebra), and "of type IV" if not, in which case $G$ is a
	 simple complex Lie group. 
\end{remark}
\begin{theorem}\label{3}
	If $(M, g)$ is a compact simply-connected irreducible
	symmetric space, then either $G$ is simple or there exists a
	real simple compact simply-connected Lie group $H$ with
	center $Z$ such that $G=(H \times H) / Z$ and $K=H / Z$,
	where $Z$ and $H$ are embedded in $H \times H$ via the
	diagonal embedding $h \mapsto(h, h)$.
\end{theorem}
\bproof
See \cite{Besse} Theorem 7.81.
\eproof
\begin{remark}
	These last symmetric spaces $H \times H / H$ are called "of
	type II". 	
\end{remark}

Now we are left with the case of compact irreducible symmetric
spaces such that $G$ is simple. These ones are called "of type
I". Another trick shows that once again the classification is a
consequence of the classification of real simple Lie groups. It
goes as follows; let $\ssg^{\mathbb{C}} = \ssg \otimes
\mathbb{C}$ be the complexification of $g$ and $G^{\bbC}$ the
simply-connected complex Lie group generated by $\ssg^{\bbC} .$
Now $\ssg_{1}=\mathfrak{k} \oplus i \ssp$ is obviously
a (real) Lie subalgebra of $\ssg^{C}$. Let $G_{1}$ be the
subgroup of $G^{\mathbb{C}}$ generated by $\ssg_{1} .$ Since $f
\subset g_{1}, G_{1}$ contains $K$ and we see easily that $G_{1}
/ K$ is a non compact irreducible symmetric space, necessarily
of type III.

Now the classification of spaces of type I follows from the
classification of real simple Lie groups and their maximal
compact subgroups.

The tables of the four types can be found in \cite{Besse} page
201-202.

\begin{remark}
	 1) More generally, the same construction works for any
	 irreducible symmetric space. The space $M_{1}=G_{1} / K$ is
	 called the "dual" of $M .$ This duality interchanges types
	 I and III, and types II and IV, respectively.
	 
	2) In summary, irreducible symmetric spaces of type I or II
	are compact, Einstein and have non-negative sectional
	curvature. Irreducible symmetric spaces of type III or IV
	are non-compact, Einstein and have non-positive sectional
	curvature.
\end{remark}
